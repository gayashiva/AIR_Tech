\documentclass[utf8]{frontiersSCNS}
\usepackage{gensymb}
\usepackage{url,hyperref,lineno,microtype,subcaption}
\usepackage[onehalfspacing]{setspace}

\linenumbers
\usepackage{wasysym} % provides \DH, \dh, \Thorn, \thorn
% Leave a blank\usepackage{amsmath}
%\DeclareMathOperator{\sign}{sign} line between paragraphs instead of using \\

\usepackage{booktabs}
\usepackage{multirow}
\usepackage{siunitx} %for SI units
\usepackage{lscape} % for landscape table

\def\keyFont{\fontsize{8}{11}\helveticabold }
\def\firstAuthorLast{Balasubramanian {et~al.}} %use et al only if is more than 1 author
\def\Authors{Suryanarayanan Balasubramanian\,$^{1}$, Roger Waser\,$^{2}$, Martin Hoelzle\,$^{1}$}
\def\Address{$^{1}$University of Fribourg, Department of Geosciences, Fribourg, Switzerland $^{2}$University of
Applied Sciences and Arts, Luzern, Switzerland} \def\corrAuthor{Suryanarayanan Balasubramanian}

\def\corrEmail{suryanarayanan.balasubramanian@unifr.ch}


\begin{document}
\onecolumn
\firstpage{1}

\title[Scheduling AIR fountains]{Fountain scheduling for efficient artificial ice reservoirs (Icestupas): theory and
practice }

\author[\firstAuthorLast ]{\Authors}
\address{}
\correspondance{}

\extraAuth{}

% \maketitle

\begin{abstract}
  To construct artificial ice reservoirs (AIRs) with limited water resources requires a high water use
  efficiency (WUE). This can be achieved by the precise scheduling of the fountain water spray taking into
  account the AIRs response to weather conditions. Particularly in the light of climate change with rising
  dependence on this water storage technology, an optimal solution for this task is of paramount importance. We
  solve the corresponding optimization problem, i.e. finding the ideal discharge rate for maximum WUE by an
  approach which offers straightforward application facilities. The automation software uses a simplified
  equation with 6 coefficients that capture the influence of temperature, humidity, wind and solar radiation
  variations on the freezing rate. Historical meteorological data of the site is required to calculate these 6
  coefficients. The automated AIR had a WUE three times more than the manual AIR. These results support the use
  of the model in practice.

	\tiny
	\keyFont{ \section{Keywords:} icestupa, water storage, climate change adaptation, geoengineering, nature based
  solution} %All article types: you may provide up to 8 keywords; at least 5 are mandatory.
\end{abstract}

\section{Introduction}
Artificial Ice Reservoirs (AIRs) have recently received much attention in the light of increasing users,
especially in semi-arid and arid mountain regions facing limited water availability and at the same time
increased water needs. However, their water use efficiency (WUE) is very poor. 

Fountain scheduling methods are one option for reducing watering volumes for AIR construction systems and, at
the same time, increase WUE. The goal of fountain scheduling is to make the most efficient use of water by
spraying the right amount of water at the right time, making sure water is available when the AIR surface can
freeze it. Scheduling maximizes AIR WUE by minimizing fountain waste water.  

However, efficient AIR construction is challenging, due to the many factors that should be considered, including
weather, water source and fountain characteristics. Moreover, due to the daily variations of the AIR freezing
freezing rate proper fountain scheduling can only be achieved through continuous adjustments of the fountain
discharge rate. 

Proper fountain scheduling requires answers to two questions: (a) When should the water be turned on and off?
(b) How much water should be sprayed? Question (a) can be answered accurately based on the sign of the surface
energy balance. However, question (b) requires information on both the magnitude of the energy balance and the
fountain spray radius over which this energy balance acts.  

Methods of fountain scheduling can be classified as static or dynamic. According to the static approach the
total amount of water for AIR construction is allocated without specifying its temporal distribution along the
accumulation period. By contrast, in the dynamic approach water is allocated at specific time steps along the
accumulation period in order to achieve even higher WUE.

\section{Study Sites and data}

\section{Automation software}
The objective of the automation software is to estimate the freezing rate given minimal weather, fountain and
location information. In order to do so, we simplify the methodology used in \cite{Balasubramanian_2022}.
Several assumptions were required to reduce the full energy balance model developed to a function with minimal
variables. We tend to use assumptions that underestimate the associated freezing rate to optimise for WUE. Below
we present each of our assumptions corresponding to the different model modules used in
\cite{Balasubramanian_2022}. 

% \begin{landscape}
% \begin{table}[]
% \centering
% \caption{}
% \label{tab:my-table}
% \begin{tabular}{@{}lllll@{}}
% \toprule
% \textbf{Module name} & \textbf{Symbol} & \textbf{Full eqn} & \textbf{Simplified eqn} & \textbf{Assumptions} \\ \midrule
% \multicolumn{1}{|l}{Surface Area}        & $A_{cone}$ &  & \pi \cdot r_{F}^2 & \multicolumn{1}{l|}{} \\ \midrule
% \multicolumn{1}{|l}{Shortwave Radiation} & $q_{SW}$ &  & (1- \alpha_{ice}) \cdot ( (1- cld) \cdot SW_{global} \cdot f_{cone} + cld \cdot SW_{global}) & \multicolumn{1}{l|}{} \\ \midrule
% \multicolumn{1}{|l}{Longwave Radiation}  & $q_{LW}$ &  & \sigma \cdot \epsilon_a \cdot {(T_a+ 273.15)}^4 -\sigma \cdot \epsilon_{ice} \cdot {273.15}^4 & \multicolumn{1}{l|}{} \\ \midrule
% \multicolumn{1}{|l}{Sensible Heat}       & $q_{S}$ &  &  & \multicolumn{1}{l|}{} \\ \midrule
% \multicolumn{1}{|l}{Latent Heat}         & $q_{L}$ &  &  & \multicolumn{1}{l|}{} \\ \bottomrule
% \multicolumn{1}{|l}{Temperature heat flux} & $q_{T}$ &  & 0 & \multicolumn{1}{l|}{} \\ \bottomrule
% \multicolumn{1}{|l}{Fountain discharge heat flux} & $q_{F}$ &  & 0 & \multicolumn{1}{l|}{} \\ \bottomrule
% \multicolumn{1}{|l}{Ground heat flux}    & $q_{G}$ &  & 0 & \multicolumn{1}{l|}{} \\ \bottomrule
% \end{tabular}
% \end{table}
% \end{landscape}

\subsection{Surface area calculation} \label{sec:shape}

The software approximates the area of the conical AIR to be equal to the area of its circular base produced
through the fountain spray radius $r_F$. Therefore, the surface area can be determined using

\begin{equation} A_{cone} =\pi \cdot r_{F}^2 \label{eq:Area} \end{equation}

Admittedly, this assumption underestimates the surface area of the AIR during the accumulation period and
thereby underestimates the freezing rate.

\subsection{Energy balance calculation} \label{sec:energy}

We approximate the energy balance at the surface of an AIR by a one-dimensional description of energy fluxes as
used in \cite{Balasubramanian_2022}:

\begin{equation}
	 q_{total} = q_{SW} + q_{LW} + q_{L} + q_{S} + q_{F} + q_{G}
	\label{eqn:EB}
\end{equation}

Upward and downward fluxes relative to the ice surface are positive and negative, respectively. The first
term represents the energy change used for freezing the fountain water and melting the ice respectively.
$q_{SW}$ is the net shortwave radiation; $q_{LW}$ is the net longwave radiation; $q_{L}$ and $q_{S}$ are the
turbulent latent and sensible heat fluxes. 

The software assumes $T_{ice} = 0 \degree C$ and therefore ignores $q_{T}$, $q_{F}$ and $q_{G}$. All these
assumptions overestimate the freezing rate.

\subsubsection{Net Shortwave Radiation \texorpdfstring{$q_{SW}$}{Lg}} \label{sec:SW}

The net shortwave radiation $q_{SW}$ is computed as follows:

\begin{equation} q_{SW} = (1- \alpha_{ice}) \cdot ( (1- cld) \cdot SW_{global} \cdot f_{cone} + cld \cdot SW_{global}) \label{eqn:SW} \end{equation}

where $\alpha_{ice}$ is the bare ice albedo value (0.25); $SW_{global}$ is the global shortwave radiation and
$cld$ is a cloudiness index determined from historical weather data. The global shortwave radiation used is modelled using the parametrisation proposed by \cite{Woolf_1968}.

The solar area fraction $f_{cone}$ of the ice structure exposed to the direct shortwave radiation depends on the
shape considered. Using the solar elevation angle $\theta_{sun}$, the solar beam can be considered to have a
vertical component, impinging on the horizontal surface (semicircular base of the AIR), and a horizontal
component impinging on the vertical cross section (a triangle). The solar elevation angle $\theta_{sun}$ used is
modelled using the parametrisation proposed by \cite{Woolf_1968}. Here we overestimate the impact of direct
solar radiation by assuming $h_{cone} = r_{cone} = r_{F}$. Accordingly, $f_{cone}$ is determined as
follows:

\begin{equation}
	\begin{split}
		f_{cone}& =\frac{ cos \theta_{sun} + \pi \cdot sin \theta_{sun} }{2\sqrt{2} \cdot \pi }\\
	\end{split}
	\label{ eqn:f_{cone} }
\end{equation}

The software ignores the variations in the albedo and assumes it to be equal to that of ice to simplify the
model. This assumption overestimates the solar radiation absorption thereby underestimating the freezing rate.

\subsubsection{Net Longwave Radiation \texorpdfstring{$q_{LW}$}{Lg}} \label{sec:LW}

The net longwave radiation $q_{LW}$ is determined as follows:

\begin{equation}
	q_{LW}= \sigma \cdot \epsilon_a \cdot {(T_a+ 273.15)}^4 -\sigma \cdot \epsilon_{ice} \cdot {(T_{ice}+ 273.15)}^4
	\label{eqn:LW}
\end{equation}

where $T_a$ represents the measured air temperature, $\epsilon_a$ denotes the atmospheric emissivity $T_{ice}$
is the modelled surface temperature given in [$\degree C$], $\sigma=5.67\cdot10^{-8}\,Jm^{-2}s^{-1}K^{-4}$ is
the Stefan-Boltzmann constant and $\epsilon_{ice}$ is the corresponding emissivity value for the Icestupa
surface (0.97).

We approximate the atmospheric emissivity $\epsilon_a$ using the equation suggested by \cite{Brutsaert_1975},
considering air temperature and vapor pressure (Eqn. \ref{eqn:atm_e}). The vapor pressure of air over water and
ice was obtained using Eqn. \ref{eqn:vp}.  The expression defined in \cite{Brutsaert_1975} for clear skies
(first term in equation \ref{eqn:atm_e}) is extended with the correction for cloudy skies after
\cite{Brutsaert_1982} as follows:

\begin{equation}
	\epsilon_a=1.24 \cdot (\frac{p_{v,w}}{(T_a+273.15)})^{1/7}\cdot(1+0.22\cdot{cld}^2) \label{eqn:atm_e}
\end{equation}

with a cloudiness index $cld$, ranging from 0 for clear skies to 1 for complete overcast skies. 

The software assumes $T_{ice} = 0 \degree C$ and ignores the influence of clouds in the atmospheric emissivity.
The first assumption overestimates $q_{LW}$ wherease the seond one underestimates it.  

\subsubsection{Turbulent fluxes} \label{sec:Qs}

The turbulent sensible $q_{S}$ and latent heat $q_{L}$ fluxes are computed with the following expressions
proposed by \cite{Garratt_1992}:

\begin{equation}
	q_{S}= c_{a} \cdot \rho_{a} \cdot p_{a}/p_{0,a} \cdot \frac{\kappa^2 \cdot v_a \cdot
		(T_a-T_{ice})}{{(\ln{\frac{h_{AWS}}{z_{0}}})}^2}
	\label{eqn:qs}
\end{equation}

\begin{equation}
	q_{L}= 0.623 \cdot L_s \cdot \rho_{a}/p_{0,a} \cdot \frac{\kappa^2 \cdot
	v_a(p_{v,w}-p_{v,ice})}{{(\ln{\frac{h_{AWS}}{z_{0}}})}^2}
\end{equation}

where $h_{AWS}$ is the measurement height above the ground surface of the AWS (around $2\,m$ for all sites),
$v_a$ is the wind speed in [$m\,s^{-1}$], $c_a$ is the specific heat of air at constant pressure (1010 J
$kg^{-1} K^{-1}$), $\rho_{a}$ is the air density at standard sea level (1.29 $kg m^{-3}$), $p_{0,a}$ is the air
pressure at standard sea level (1013 $hPa$), $p_{a}$ is the measured air pressure, $\kappa$ is the von Karman
constant (0.4), $z_{0}$ is the surface roughness (3 $mm$) and $L_s$ is the heat of sublimation (2848
$kJ\,kg^{-1}$).  The vapor pressure of air with respect to water ($p_{v,w}$) and with respect to ice
($p_{v,ice}$) was obtained using the formulation given in \cite{huang_2018} :

\begin{equation}
	\begin{split}
		p_{v,w}&=e^{\frac{(34.494 - \frac{4924.99}{T_{a} + 237.1})}{(T_a + 105)^{1.57} \cdot 100}} \cdot \frac{RH}{100} \\
		p_{v,ice}&=e^{\frac{(43.494 - \frac{6545.89}{T_{ice} + 278})}{(T_{ice} + 868)^{2} \cdot 100}} \\
	\end{split} \label{eqn:vp}
\end{equation}

The software ignores the $\mu_{cone}$ parameter thereby underestimating the turbulent fluxes.

\subsection{Automation equation}

Using the above simplifications, we can now approximate $q_{LW}$, $q_{L}$ and $q_{S}$ using just temperature,
relative humidity and wind speed. 

Similarly, the diurnal variation of the shortwave radiation can be captured via a gaussian equation. However,
the seasonal variations in the amplitude of the solar radiation are poorly captured by such an equation. Since,
typical AIRs have a construction period spanning less than 3 months, we can ignore the seasonal variations of
solar radiation. Therefore, the net shortwave radiation $q_{SW}$ is computed as follows: 


\begin{subequations}

	\begin{align}
		\label{eqn:SW}
    \frac{q_{SW} \cdot A_{cone} \cdot \Delta t}{L_F} & = \frac{amp}{(\sigma \sqrt{2\pi})} \cdot
    exp\left(\frac{-(time-\mu)^2}{2\sigma^2}\right) \\
		\label{eqn:T}
    \frac{(q_{LW} + q_{S} + q_{L}) \cdot A_{cone} \cdot \Delta t}{L_F} & = a \cdot T_a + b \cdot RH + c \cdot v_a +
  d \\
		\label{eqn:auto}
    \frac{q_{total} \cdot A_{cone} \cdot \Delta t}{L_F} & = \frac{amp}{(\sigma \sqrt{2\pi})} \cdot
    exp\left(\frac{-(time-\mu)^2}{2\sigma^2}\right) + a \cdot T_a + b \cdot RH + c \cdot v_a + d
	\end{align}
\end{subequations}

The first term in Eqn. \ref{eqn:auto} is the expected freezing/melting rate of the AIR. Hence the automation
equation can now be formulated as:

\begin{subequations}
	\begin{align}
		\label{eqn:SW}
  \frac{q_{total} \cdot A_{cone} \cdot \Delta t}{L_F} & = \left\{ \begin{array}{ll}
		\frac{\Delta M_{water}}{\Delta t} & \textit{ if } q_{total} > 0 \\
		\frac{-\Delta M_{ice}}{\Delta t} & \textit{ if } q_{total} \leq 0
	\end{array} \right. \\
  f(time;T_a;RH;v_a) & = \left\{ \begin{array}{ll}
		\frac{amp}{(\sigma \sqrt{2\pi})} \cdot
    exp\left(\frac{-(time-\mu)^2}{2\sigma^2}\right) + a \cdot T_a + b \cdot RH + c \cdot v_a + d
    & \textit{ if } \frac{\Delta M_{ice}}{\Delta t} > d_{crit} \\
		0 & \textit{ if } \frac{\Delta M_{ice}}{\Delta t} \leq d_{crit}
	\end{array} \right.
	\end{align}
\end{subequations}


, the automation equation is composed of two parts namely, (a) linear and (b) gaussian equation. (a)
approximates the influence of temperature, wind speed and humidity on the expected freezing rate. (b)
approximates the contribution of solar radiation on the expected freezing rates.

\section{Automation hardware}

\section{Results}

% \begin{figure}
% 	\begin{center}
% 		\includegraphics[width=\linewidth]{Figures/autovsmanual.jpg}
% 	\end{center}
% 	\caption{Icestupa in Ladakh, India on March 2017 was 24 $m$ tall and contained around 3700 $m^3$
% 		of water. Picture Credits: Lobzang Dadul}
% 	\label{fig:old_icestupa}
% \end{figure}

\subsection{Validation}

\section{Discussion}

\section{Conclusions}

\section{Appendix}

\bibliographystyle{frontiersinSCNS_ENG_HUMS} \bibliography{refs}

\end{document}
