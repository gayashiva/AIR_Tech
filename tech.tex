\documentclass[tc, manuscript]{copernicus}

\usepackage{booktabs}
\usepackage{multirow}
\usepackage{siunitx} 

\begin{document}

\title{Improving water-use efficiency of artificial ice reservoirs (Icestupas) through fountain scheduling
strategies}

\def\Authors{Suryanarayanan Balasubramanian\,$^{1,2}$, Martin Hoelzle\,$^{1}$Roger Waser\,$^{3}$}

\def\Address{$^{1}$University of Fribourg, Department of Geosciences, Fribourg, Switzerland $^{2}$University of
Applied Sciences and Arts, Luzern, Switzerland} \def\corrAuthor{Suryanarayanan Balasubramanian}
\Author[1,2]{Suryanarayanan}{Balasubramanian}
\Author[1]{Martin}{Hoelzle}
\Author[3]{Roger}{Waser}
\affil[1]{University of Fribourg, Department of Geosciences, Fribourg, Switzerland}
\affil[2]{Himalayan Institute of Alternatives, Ladakh, India}
\affil[3]{University of Applied Sciences and Arts, Luzern, Switzerland}

\correspondence{suryanarayanan.balasubramanian@unifr.ch}

\runningtitle{Scheduling AIR fountains}

\runningauthor{S. Balasubramanian}

\firstpage{1}

\maketitle

\begin{abstract}

  Artificial Ice Reservoirs (AIRs), often also called - Ice Stupas - are a climate change adaptation strategy
  developed in the Indian Himalayas (Ladakh). With this technology, otherwise unused stream water is stored in
  large ice towers in winter. The surplus melt water that is generated in spring is used for satisfying
  irrigation water demands. Recent studies have shown that during construction of traditional AIRs over 75 \% of
  the water sprayed was lost. Therefore, fountain wastewater production has to be reduced for improving water
  use efficiency.  During the winter of 2021-22, a traditional and an automated AIR were built in Guttannen,
  Canton of Berne, Switzerland with the main aim of comparing and quantifying the benefits of fountain
  scheduling. Fountain scheduling was realized through an automation system computing recommended discharge
  rates using real-time weather input and location metadata. The scheduled fountain produced similar volumes
  while consuming 87 \% less water than the unscheduled fountain. Simulations converting unscheduled fountains
  to scheduled fountains improved the water use efficiency of several traditional AIRs more than two fold.
  Overall, these results show that the automated construction strategy can increase the water use efficiency of
  AIRs without compromising their meltwater production.

\end{abstract}

\introduction

Cryosphere-fed irrigation networks in arid mountain regions are completely dependent on timely availability of
meltwater from glaciers, snow and permafrost \citep{immerzeelImportanceVulnerabilityWorld2020, farhanHydrologicalRegimesConjunction2015,
tveitenGlacierGrowingLocal2007}. With the accelerated decline of glaciers due to climate change
\citep{nusserLocalKnowledgeGlobal2016}, these regions are experiencing water scarcity particularly during the
spring season \citep{norphelSnowWaterHarvesting2015}. This seasonal water scarcity makes it essential to provide
supplementary irrigation in order to sustain agricultural output and take advantage of the complete growing
season \citep{nusserLocalKnowledgeGlobal2016, vincentEnergyClimateChange2009}.

To cope with this recurrent water scarcity, villagers in the region of Ladakh have developed two types of
artificial ice reservoirs (AIRs): ice terraces and ice stupas (see Fig. \ref{fig:AIRforms}).  All these types of
ice reservoirs capture water in the autumn and winter, allowing it to freeze, and hold it until spring, when it
melts and flows down to fields \citep{ipccChapterHighMountain2019, vinceGlacierMan2009,
clouseLadakhArtificialGlaciers2017, nusserSociohydrologyArtificialGlaciers2019}. In this way, they retain a
previously unused portion of the annual flow and facilitate its use to supplement the decreased flow in the
following spring. This study focuses on the form of AIRs locally called as ice stupas.

\begin{figure}[t]
\includegraphics[width=12cm]{Figures/AIR_forms.jpg}

\caption{(a) Schematic overview of the position of artificial ice reservoirs. These constructions are located at
altitudes between the glaciers and the irrigation networks in the cultivated areas. Ice terraces and ice stupas
are located at higher and lower altitudes respectively. Adapted from: \cite{nusserLocalKnowledgeGlobal2016}}

\label{fig:AIRforms}
\end{figure}

Over the past decade, several ice stupas have been built to supplement irrigation water supply of mountain
villages in India \citep{wangchukIceStupaCompetition2020, palmerStoringFrozenWater2022,
aggarwalAdaptationClimateChange2021}, Kyrgyzstan \citep{bbcnewsBrightArtificialGlacier2020} and Chile
\citep{reutersConservationistsChileAim2021}. These AIRs are traditionally constructed by diverting springs or
glacial streams into fountain spray systems via embankments and pipelines. 

One of the most common problems with AIR construction systems has to do with fountain scheduling. Fountain
scheduling is simply answering the questions of “When do we spray?”, “How much do we spray?” and “How long do we
spray?”. Starting fountain spray too early or spraying too much water or running a fountain spray too long is
considered overwatering. At the very least this practice wastes water. However, overwatering can cause
accelerated ice melt if done on a prolonged basis. Likewise, starting fountain spray too late or spraying too
little water or not running the system for a long enough period of time is considered under watering and can
cause reduced ice volumes.

Previous work \citep{balasubramanianInfluenceMeteorologicalConditions2022} has shown that traditional
construction systems suffer from overwatering. Knowledge of surface freezing rates is important to
avoid overwatering. Surface freezing rates can be calculated by means of the full energy balance model
developed in \cite{balasubramanianInfluenceMeteorologicalConditions2022}. This model can be forced with either
historical weather data or real-time weather data to produce recommended discharge rates.

However, there are several issues that need to be addressed before operating such weather-sensitive fountains.
For example, in the case of the Indian AIR, the fountain discharge rate could have been halved since they were
always two times higher than the modelled freezing rate
\citep{balasubramanianInfluenceMeteorologicalConditions2022}. However, in practice, reduction of discharge rate
would increase the maintenance cost due to higher risk of fountain freezing events.

An optimum construction strategy, therefore, should first prevent the occurrence of this event. These fountain
freezing events can be prevented by setting a minimum threshold for the recommended discharge rate.
Additionally, recommended discharge rate needs to be sensitive to constraints on the water supply or weather of
the construction site. Locations limited by their water supply like in Ladakh, India would prioritize water use
efficiency whereas those limited by the favourable weather windows like in Guttannen, Switzerland  would
prioritize for maximum ice volume.  Accordingly, we use two types of model parameter optimizations that prevent
underwatering and overwatering to attain higher ice volumes and higher water use efficiency respectively.

However, manually adjusting the fountain discharge rate is not practical due to two reasons. Firstly, this would
involve constant adjustments of discharge rates in response to the significant diurnal and seasonal variations
of the freezing rates. Secondly, frequent pipeline water drainage is required to avoid water losses. Therefore,
operation of scheduled fountains via automation systems is preferred to reduce the long-term maintenance costs.

The present study was performed to compare two AIRs produced using different fountain scheduling strategies but
exposed to identical meteorological conditions. The specific objectives of this study is to compare the
water-use efficiency, maximum ice volume and maintenance effort between the different fountain scheduling strategies.

\section{Study sites and data}

In this study, we use some datasets presented in our previous work
\citep{balasubramanianInfluenceMeteorologicalConditions2022} along with new datasets. These old datasets record
the meteorological conditions and fountain characteristics of AIRs built in Gangles, India (IN21) and Guttannen,
Switzerland (CH21) during the winter of 2020-21. In this section, we focus on describing the new AIR datasets
collected in Guttannen, Switzerland during the winter of 2021-22 (CH22).

The Guttannen site (46.66 $\degree$N, 8.29 $\degree$E) is situated in the Berne region, Switzerland and has an
altitude of 1047 $m$ a.s.l. In the winter (Oct-Apr), mean daily minimum and maximum air temperatures vary
between -13 and 15 $\degree C$. Clear skies are rare, averaging around 7 days during winter. Daily winter
precipitation can sometimes be as high as 100 $mm$. These values are based on 30 years of hourly historical
weather data measurements \citep{meteoblueClimateGuttannen2021}. Two AIRs were constructed by the Guttannen
Bewegt Association, the University of Fribourg and the Lucerne University of Applied Sciences and Arts during
the winters of 2021-22 using a traditional and an automated construction strategy.

\begin{figure}[t]
\includegraphics[width=12cm]{Figures/AIR_fountains.jpg}
\caption{Unscheduled and scheduled fountains used for construction of traditional an automated AIRs at Guttannen. Picture credits: Daniel Bürki}
\label{fig:2AIR}
\end{figure}

The automated and the traditional AIRs were constructed adjacent to each other but with different fountain
characteristics as shown in Fig. \ref{fig:2AIR}. This ensured both AIRs shared the same water source and
identical weather conditions. In addition, a webcam guaranteed a continuous survey of the automated AIR.   

In the traditional strategy, the fountain was operated manually whereas in the automated strategy a programmed
automation system controlled the fountain discharge rate during the whole study period using real time weather
input and several control parameters, which could be modified via a user interface. Henceforth, we refer to the
fountain used in the traditional and automated construction strategy as unscheduled and scheduled fountains
respectively.

In the traditional construction strategy, tree branches were laid covering the fountain pipe to initiate and
speed up the ice cone formation process. In the automated strategy, only the fountain pipe was placed before the
water spray started. The construction of both the AIRs began on 8th December on a snow bed of 13 cm thickness
and ended on 12th April. These two dates are denoted as start and expiry dates henceforth.

\subsection{Meteorological data}

Air temperature, relative humidity, wind speed, pressure, long-wave, precipitation, global short-wave radiation
and cloudiness index are required to calculate the surface energy balance of an AIR. The primary weather data
source was an automatic weather station (AWS) located around 20 m away. Hourly ground temperature measurements
were also recorded by the AWS to approximate the fountain's water temperature. Less than 0.4 \% of the data was
found to be missing and the data gaps were filled by linear interpolation. However, two additional datasets were
required to obtain all the necessary input variables, namely, cloudiness index and precipitation. These two
datasets were obtained from ERA5 reanalysis dataset \citep{hersbachERA5GlobalReanalysis2020} and a meteoswiss AWS
located 184 m away (Station ID: 0-0756-0-GTT) respectively.

\begin{figure*}[t]
\includegraphics[width=12cm]{Figures/disvstemp.png}
\caption{Temperature and discharge measurements at the Guttannen construction site}
\label{fig:aws} 
\end{figure*}

\subsection{Fountain observations}

The scheduled and unscheduled fountains have three attributes, namely: discharge rate ($Q$), height ($h$) and
water temperature ($T_F$). Discharge rate represents the discharge rate of the water in the fountain pipeline.
Height denotes the height of the fountain pipeline installed. Fountain water temperature is the temperature of
water droplets produced by the fountain.

The height was increased in steps of 1 meter for both the fountains. For the scheduled fountain, the
construction began with a height of 3 m with an increase to 4 $m$ on 23rd December. For the unscheduled
fountain, the construction began with a height of 3.7 $m$ and then this was increased two times on 23rd December
and 12th February. The variation in the discharge rate due to these height increase events can b observed in
Fig. \ref{fig:aws}. 

Fig. \ref{fig:aws} shows how the automation system operated the scheduled fountain based on real-time
meteorological conditions. The automation system caused these variations through its control valve which varied
between 0 to 100 \% depending on the recommended discharge rate. Throughout the study period, the control valve
was opened completely (100 \%) only once corresponding to the time when the temperature attained its minimum of
-13 \degree C on 20th December. After this event, the control valve was never opened beyond 34 \%.  

The unscheduled fountain was manually operated to spray all the available discharge until a fountain freezing
event interrupted the discharge on 17th February. Unfortunately, no discharge rate measurements were recorded
for the unscheduled fountain. However, the unscheduled fountain was observed to have a higher discharge rate
compared to the scheduled fountain due to its higher aperture area (see Fig. \ref{fig:2AIR}). Therefore, we
conservatively assume the discharge rate of the unscheduled fountain to be equal to the maximum discharge rate
of the scheduled fountain which was observed to be 13 $l/min$ at a fountain height of 3 $m$ and 11 $l/min$ at a
fountain height of 4 $m$ respectively.

The water temperature of both the fountains were estimated from the AWS ground temperature dataset.

\subsection{Drone surveys}

Several photogrammetric surveys were conducted on the traditional and the automated AIRs. The details of these
surveys and the methodology used to produce the corresponding outputs are explained in
\cite{balasubramanianInfluenceMeteorologicalConditions2022}. The digital elevation models (DEMs) generated from
the obtained imagery were analysed to document the ice radius, the surface area and the volume of the ice
structures. Ice radius measurements of drone flights which observed either an increase in AIR circumference or
volume were averaged to determine the fountain's spray radius. The number of drone surveys conducted for the
traditional and the automated AIRs were 8 and 6, respectively (see Table \ref{tab:uav}). 

\begin{table}
	\centering
	\caption{ Summary of the drone surveys}
	\label{tab:uav}
	\begin{tabular}{@{}|llllll|@{}}
		\toprule
		\textbf{}              & \textbf{No.} & \textbf{Date} & \textbf{Volume} & \textbf{Radius} & \textbf{Surface Area} \\ \midrule
		\multicolumn{1}{|l|}{\multirow{8}{*}{\rotatebox[origin=c]{90}{Traditional}}}
		                       & 1            & Dec 23, 2021  & 17 $m^{3}$     & 2.9 $m$
		                       & 47 $m^{2}$                                                                      \\
		\multicolumn{1}{|l|}{} & 2            & Jan 3, 2022  & 22 $m^{3}$     & 3.4 $m$
		                       & 61 $m^{2}$                                                                      \\
		\multicolumn{1}{|l|}{} & 3            & Jan 22, 2022   & 35 $m^{3}$     & 4 $m$
		                       & 79 $m^{2}$                                                                      \\
		\multicolumn{1}{|l|}{} & 4            & Feb 6, 2022  & 44 $m^{3}$     & 4.2 $m$
		                       & 86 $m^{2}$                                                                      \\
		\multicolumn{1}{|l|}{} & 5            & Feb 20, 2022  & 43 $m^{3}$     & 4.3 $m$
		                       & 86 $m^{2}$                                                                      \\
		\multicolumn{1}{|l|}{} & 6            & Mar 19, 2022  & 33 $m^{3}$     & 4.4 $m$
		                       & 84 $m^{2}$                                                                      \\
		\multicolumn{1}{|l|}{} & 7            & Mar 26, 2022  & 24 $m^{3}$     & 4.3 $m$
		                       & 74 $m^{2}$                                                                      \\
		\multicolumn{1}{|l|}{} & 8            & Apr 12, 2022  & 11 $m^{3}$     & 3.5 $m$
		                       & 50 $m^{2}$                                                                      
		\\\midrule
		\multicolumn{1}{|l|}{\multirow{6}{*}{\rotatebox[origin=c]{90}{Automated}}}
		                       & 1            & Dec 23, 2021  & 35 $m^{3}$      & 4.3 $m$
		                       & 73 $m^{2}$                                                                       \\
		\multicolumn{1}{|l|}{} & 2            & Jan 3, 2022   & 32 $m^{3}$      & 4.4 $m$
		                       & 81 $m^{2}$                                                                       \\
		\multicolumn{1}{|l|}{} & 3            & Feb 20, 2022   & 60 $m^{3}$      & 5.3 $m$
		                       & 105 $m^{2}$                                                                       \\
		\multicolumn{1}{|l|}{} & 4            & Mar 19, 2022   & 28 $m^{3}$      & 3.7 $m$
		                       & 57 $m^{2}$                                                                       \\
		\multicolumn{1}{|l|}{} & 5            & Mar 26, 2022   & 19 $m^{3}$      & 3.7 $m$
		                       & 53 $m^{2}$                                                                       \\
		\multicolumn{1}{|l|}{} & 6            & Apr 12, 2022   & 7 $m^{3}$      & 2.5 $m$
		                       & 53 $m^{2}$                                                                       \\
		\bottomrule
	\end{tabular}

\end{table}

\section{Methods}

\subsection{Determination of recommended discharge rate}

Locations constrained by their available water supply require water-sensitive fountain scheduling strategies and
those constrained by the duration of favourable weather windows require weather-sensitive fountain scheduling
strategies. These two kinds of scheduled fountains will be referred to as water-sensitive fountain and
weather-sensitive fountain henceforth.

Water-sensitive fountains are expected to produce higher water use effiency whereas weather-sensitive fountains
are expected to produced higher ice volumes. Accordingly, we create two sets of model forcing assumptions to
produce the discharge rate recommendations for these two kinds of fountains for the following three model
variables: (a) slope , (b) albedo and (c) cloudiness.  These two kinds of models will be referred to as ice
volume optimised model (IVOM) and water-use efficiency optimised model (WEOM) respectively. The slope variable
increases the shortwave radiation and sensible heat impact. The albedo variable decreases the shortwave
radiation impact. The cloudiness variable increases both the shortwave and the longwave radiation impact. We
associate the upper and lower bounds of these variables with the IVOM and WEOM versions depending on whether
they overestimate and underestimate the freezing rate of the AIR as shown in Table \ref{tab:assumptions}.

\begin{table}[]
\centering
\caption{Assumptions for the parametrisation introduced to simplify the model.}
\label{tab:assumptions}
\begin{tabular}{@{}lllll@{}}
\toprule
\textbf{Estimation of} & \textbf{Symbol} & \textbf{IVOM} & \textbf{WEOM} & \\ \midrule
\multicolumn{1}{|l}{Slope}        & $s_{cone}$ & $ 1 $ & $0$ & \multicolumn{1}{l|}{} \\ \midrule
\multicolumn{1}{|l}{Albedo} & $\alpha$ & $\alpha_{snow}$ & $\alpha_{ice}$ & \multicolumn{1}{l|}{} \\\midrule 
\multicolumn{1}{|l}{Cloudiness}  & $cld$ & $0$ & $1$ & \multicolumn{1}{l|}{} \\ \bottomrule
\end{tabular}
\end{table}

We apply the assumptions described in Table \ref{tab:assumptions} on the one-dimensional description of energy
fluxes as used in \cite{balasubramanianInfluenceMeteorologicalConditions2022} to obtain the rate of change of
AIR ice mass as follows: 

\begin{equation}
  \frac{\Delta M_{ice}}{\Delta t}  =  (\frac{q_{SW} + q_{LW} + q_{S} + q_{F} + q_{R} + q_{G} - q_{T}}{L_F} + \frac{q_{L}}{L_V} ) \cdot A_{cone}
	\label{eqn:auto}
\end{equation}

Upward and downward fluxes relative to the ice surface are positive and negative, respectively. The first term
represents the mass change rate due to freezing of the fountain water and melting of the ice. $q_{SW}$ is the
net short-wave radiation; $q_{LW}$ is the net long-wave radiation; $q_{L}$ and $q_{S}$ are the turbulent latent
and sensible heat fluxes; $q_{F}$ is the fountain discharge heat flux; $q_{R}$ is the rain water heat flux;
$q_{G}$ is the ground heat flux; $q_{T}$ is the temperature heat flux and $A_{cone}$ is the area of the AIR
surface. The derivation of these individual terms for the IVOM and WEOM model versions are discussed in the
Appendix \ref{sec:SEB}.

Equation \ref{eqn:auto} is implemented in the automation system through a user interface that enables input of
the spray radius, altitude, latitude and longitude of the construction location. Once switched on, the
automation system regulates the fountain discharge rate based on the recommended discharge rate. Further details
about how the automation system functions can be found in Appendix \ref{sec:auto_system}. 


\subsection{Model updates}

The model was made more sensitive to fountain discharge rate feedback for comparing the volume evolution of the
automated and traditional AIRs. 

In the previous version of the model \citep{balasubramanianInfluenceMeteorologicalConditions2022}, the fountain
water temperature ($T_F$) was estimated as a constant parameter. However, in reality, this is a poor
approximation since it doesn't account for two processes , namely, (a) temperature fluctuations during transit
from the source to the fountain nozzle; (b) temperature fluctuations during the flight time of the water
droplets after they leave the fountain nozzle. Therefore, we instead use measured hourly ground temperature
measurements to approximate process (a) and assume water temperature cools down to 0 $\degree\,C$ during subzero
air temperature conditions to approximate process (b).

In the previous version of the model \citep{balasubramanianInfluenceMeteorologicalConditions2022}, fountain
discharge events were reset from surface albedo to ice albedo. However, this assumption limits the accuracy of
the model, especially, for the automated AIR where several fountain discharge events of short durations occur.
Therefore, we assumed that discharge events instead reduce the albedo decay rate ($\tau$) by a 
factor of $\frac{\alpha_{ice}}{\alpha_{snow}}$.

Additionally, both the AIRs experienced many precipitation events. Therefore, it was no longer accurate to
assume AIR density ($\rho_{cone}$) to be equal to ice density. We instead parameterised AIR density $\rho_{cone}$ as follows:

\begin{equation}
  \rho_{cone} = \frac{M_{F} + M_{dep} + M_{ppt}}{(M_{F} + M_{dep})/\rho_{ice} + M_{ppt}/\rho_{snow}}
\end{equation}

where $M_F$ is the cumulative mass of the fountain discharge; $M_{ppt}$ is the cumulative precipitation;
$M_{dep}$ is the cumulative accumulation through water vapour deposition; $\rho_{ice}$ is the ice density (917
$kg\,m^{-3}$) and $\rho_{snow}$ is the density of wet snow (300 $kg\,m^{-3}$) taken from
\cite{cuffeyPhysicsGlaciers2010} .

Rain events were not considered in the previous version of the model but they occur in our experiment. The
influence of rain events on the albedo and the energy balance was assumed to be similar to that of discharge
events. However, the water temperature of a rain event was assumed to be equal to the air temperature.
Accordingly, the heat flux generated due to a rain event was equal to:

\begin{equation}
  q_{R} = \frac{\Delta M_{ppt} \cdot c_{water} \cdot T_{a}}{\Delta t \cdot A_{cone}}
\end{equation}

\subsection{Calibration and Validation}

The model parameters were calibrated to the median values of the ranges presented in Appendix Table
\ref{tab:parameters}. However, the surface layer thickness parameter was calibrated to a value of 9 $cm$ for the
automated AIR instead of the default value of 5 $cm$. This calibration was necessary to prevent hourly surface
temperature fluctuations to assume unphysical values above 40 $\degree\,C$.

We performed the validation of the model on the traditional and automated AIRs by evaluating the root mean
squared error (RMSE) between volume estimates and measurements. 

The performance of the IVOM and WEOM versions of the physical model was assessed by comparing correlation of its
discharge rate estimates with the validated freezing rate of the traditional AIR.

\section{Results}

\subsection{Scheduled discharge rate simulations}

The water-sensitive and weather-sensitive fountains estimated the freezing rate of the unscheduled fountain with
a correlation around 0.4 and a RMSE less than 0.8 $l/min$ and 1.8  $l/min$ respectively. The weather-sensitive
fountain overestimated the freezing rate 93 \% of the construction duration whereas the water-sensitive fountain
overestimated the freezing rate 70 \% of the unscheduled fountain spray duration as illustrated by Fig.
\ref{fig:simvsreal}. Therefore, the IVOM model forcing was successful in prioritizing the maximum ice volume but
the WEOM model forcing could not optimize for water use efficiency. However, this large overestimation could be
due to the depression of the unscheduled fountain's freezing rate caused by its excess discharge rate. 

\begin{figure*}[t]
\includegraphics[width=8cm]{Figures/simvsreal.jpg}

\caption{ Comparison of the freezing rate estimated for the unscheduled fountain and the discharge rate of the
scheduled fountains. }

\label{fig:simvsreal}
\end{figure*}

\subsection{Model validation}

The volume estimation for the automated and traditional AIR had an RMSE of 8 $m^3$ and 6 $m^3$ with the drone
volume observations, respectively. This RMSE error is within 13 \% and 11 \% of the maximum volume of the
automated and the traditional AIR respectively. The estimated and measured AIR volumes are shown in Fig.
\ref{fig:validation}.  

\begin{figure*}[t] \includegraphics[width=12cm] {Figures/validation.png} \caption{Volume validation of the
scheduled and unscheduled fountain construction strategies.} \label{fig:validation} \end{figure*}

\begin{figure*}[t]
\includegraphics[width=12cm]{Figures/dis_processes.png}
\caption{(a) Surface albedo  and (b) fountain discharge heat flux showed significant variations between the two
  AIRS due to the differences in their discharge rates.}
\label{fig:dis_processes}
\end{figure*}

\begin{table}
	\centering
	\caption{Summary of the mass balance, energy balance, fountain and AIR characteristics estimated at the end of the respective
  simulation duration for the automated and the traditional AIRs}
	\label{tab:mb}
	\begin{tabular}{@{}|llllll|@{}}
		\toprule
		\textbf{}              & \textbf{Name}                   & \textbf{Symbol} & \textbf{Traditional} & \textbf{Automated} &
		\textbf{Units}                                                                                                       \\ \midrule
		\multicolumn{1}{|l|}{\multirow{3}{*}{\rotatebox[origin=c]{90}{Input}}}
		                       & Fountain discharge              & $M_F$           & \num{1.1e6}   & \num{1.5e5}     & $kg$  \\
		\multicolumn{1}{|l|}{} & Snowfall                        & $M_{ppt}$       & \num{9.2e3}   & \num{1.4e4}   & $kg$  \\
		\multicolumn{1}{|l|}{} & Deposition                      & $M_{dep}$       & \num{4.0e2}   & \num{4.5e2}     & $kg$  \\ \midrule
		\multicolumn{1}{|l|}{\multirow{4}{*}{\rotatebox[origin=c]{90}{Output}}}
		                       & Meltwater                       & $M_{water}$     & \num{4.5e4} & \num{5.4e4}   & $kg$  \\
		\multicolumn{1}{|l|}{} & Ice                             & $M_{ice}$       & \num{7.4e3} & \num{6.1e3}    & $kg$  \\
		\multicolumn{1}{|l|}{} & Sublimation                     & $M_{sub}$       & \num{3.7e3} & \num{4.5e3}     & $kg$  \\
		\multicolumn{1}{|l|}{} & Fountain wastewater             & $M_{waste}$     & \num{1.07e6} & \num{1.0e5}     & $kg$  \\ \midrule
		\multicolumn{1}{|l|}{\multirow{7}{*}{\rotatebox[origin=c]{90}{Energy Flux}}}
                           & Shortwave radiation             &  $q_{SW}$       & $14$  & $21$ & \% \\
		\multicolumn{1}{|l|}{} & Longwave radiation              &  $q_{LW}$       & $25$  & $25$ & \% \\
		\multicolumn{1}{|l|}{} & Sensible heat                   &  $q_{S}$        & $38$   & $33$ & \% \\
		\multicolumn{1}{|l|}{} & Latent heat                     &  $q_{L}$        & $19$  & $19$ & \% \\
		\multicolumn{1}{|l|}{} & Fountain discharge heat         &  $q_{F}$        & $4$  & $0$     & \% \\
		\multicolumn{1}{|l|}{} & Rain heat                       &  $q_{R}$        & $0$  & $0$     & \% \\
		\multicolumn{1}{|l|}{} & Ground heat                     &  $q_{G}$        & $1$   & $1$     & \% \\\midrule
		\multicolumn{1}{|l|}{\multirow{2}{*}{\rotatebox[origin=c]{90}{AIR}}}

		                       & Maximum AIR Volume              &                 & 53            & 61            & $m^{3}$ \\
		\multicolumn{1}{|l|}{} & Water Use Efficiency            &                 & 4             & 35            & \% \\\midrule
	\end{tabular}
\end{table}

\subsection{Comparison of AIR construction strategies}

Table \ref{tab:mb} shows how the different fountain scheduling strategies influence the mass and energy balance
of the respective AIR. The difference between the fountain discharge input and fountain wastewater output of the
unscheduled and scheduled fountains was around an order of magnitude. 

During fountain spray, the AIR surface (a) albedo dampens to ice albedo and (b) absorbs the heat energy of the
fountain water droplets. These two processes are the cause of the difference in the mass and energy balance
distribution shown in Table \ref{tab:mb}. The temporal variation of the magnitude of these processes are shown
in Fig. \ref{fig:dis_processes}.

The overall impact of the radiation fluxes (long-wave and short-wave) and the turbulent fluxes (sensible and
latent) on the freezing and melting energies is determined from their respective energy turnover. The energy
turnover is calculated as the sum of energy fluxes in absolute values (see Table \ref{tab:mb}). 

There is a considerable difference in the contribution of the shortwave radiation due to the effect of process
(a). Even though the unscheduled fountain was active for a much longer duration, the frequent snowfall events
counteracted the albedo feedback of its fountain discharge. In contrast, the albedo of the automated AIR was
significantly impacted by late fountain spray events particularly in the month of March and April as shown in
Fig. \ref{fig:dis_processes}. These poorly timed fountain spray events occurred because the global solar
radiation diurnal variation of the automation system is calibrated based on values for the month of February.
Therefore, poor calibration of the automation system resulted in an increased impact of shortwave radiation on
the automated AIR.

Similarly, the fountain discharge heat flux for the traditional AIR was enhanced due to process (b). The higher
discharge quantity of the unscheduled fountains and its longer duration were responsible for the significant
contribution of fountain discharge heat flux in the overall energy turnover.

\section{Discussion}

\subsection{Benefits of scheduling fountains}

The difference in water-use efficiency and maximum volume between unscheduled and scheduled fountains in the two
locations across two winters are presented in Fig. \ref{fig:wue} (a). Four experimental values (highlighted by
circles) are shown together with five simulated values (highlighted by squares).  The experimental values were
taken from the IN21 and CH21 AIRs studied in \citet{balasubramanianInfluenceMeteorologicalConditions2022} and
the CH22 AIR presented in this study. 

\begin{figure*}[t]
\includegraphics[width=\textwidth]{Figures/wue.png}

\caption{(a) The maximum volumes and water-use efficiency estimated for AIRs constructed in different locations
(represented by colours) with different fountain scheduling strategies (represented by symbols). Experimental
values are highlighted by circles and simulated values are highlighted by squares. (b) Comparison of
the unscheduled and scheduled fountain's discharge rates at the IN21 location.}

\label{fig:wue}
\end{figure*}

The water-use efficiency of all the unscheduled fountains are below 20 \%. In general, the water-use efficiency
increases more than two fold when the weather-sensitive or water-sensitive fountain is used in both the locations.  

For the Indian location, the three kinds of fountains yield significantly different results.  The discharge
duration and the max discharge rate of the three IN21 fountains were responsible for these different results
(see Fig. \ref{fig:wue} (b)). The max discharge rate of the unscheduled fountain was more than twice that of
scheduled fountains resulting in a high water loss.  Fountain freezing events caused frequent interruptions in
the unscheduled discharge rate (see Fig. \ref{fig:wue} (b)). Therefore, the discharge duration of the
unscheduled fountain was much lower resulting in lower ice volumes. The water-sensitive fountain underestimated
the freezing rate during the construction period and therefore produced much lower ice volumes compared to the
weather-sensitive fountain.

For the Swiss locations, scheduled fountains yield better water-use efficiency but do not alter the maximum
volume obtained significantly. 

\subsection{Challenges of scheduling fountains}

\subsubsection{Determination of minimum discharge rate}

The value of minimum discharge rate determines the duration of the favourable weather windows and the risk of
fountain freezing events. Therefore, its accurate quantification is critical. The risk of a fountain freezing
event is inversely proportional to the spray radius. Therefore, minimum discharge rate can be determined based
on the minimum fountain spray radius desired. In the case of the CH22 experiment, the minimum discharge rate of
2 $l/min$ corresponded to a spray radius of around 1 $m$.

\subsubsection{Determination of fountain spray radius}

To schedule fountains, one needs to estimate their spray radius accurately since the recommended discharge rate
is proportional to the square of the spray radius used. However, fountain spray radius can be influenced by both
the meteorological conditions of the site and the engineering design of the fountain used. The scheduled and the
unscheduled fountains had completely different engineering designs (see Fig. \ref{fig:2AIR}) but produced AIRs
with spray radius within a meter of each other. In contrast, the unscheduled fountain of the CH21 experiment had
a spray radius more than 2 $m$ higher than the CH22 experiment. This indicates that meteorological conditions
are a major driver of the fountain spray radius. Manual measurements of the fountain spray revealed a
maximum spray radius of just 3 m but drone measurements reveal a fountain spray radius of 4.8 m for the
scheduled fountain. This discrepancy could be caused by wind drift of water droplets. 

To validate this hypothesis, we model the projectile motion of scheduled fountain water droplets with wind speed
values taken from CH22 and CH21 experiments respectively. The details about the methodology used to do this is
explained in Appendix \ref{sec:wind}. Fig. \ref{fig:wind} shows the modelled spray radius produced using these
two wind datasets and compares them with the measured spray radius values. As illustrated, wind speed drives the
temporal variation in the spray radius. Moreover, the spray radius of the scheduled fountain with CH22 wind
values is much higher than when using CH21 wind values. Therefore, estimation of fountain spray radius is
possible only if the expected temporal variation in wind speed during the construction period can be quantified.   

\begin{figure*}[t]
\includegraphics[width=12 cm]{Figures/radf.png}
\caption{Modelled spray radius using wind values from CH22 and CH21 experiments. Measured spray radius are
indicated as dots.}
\label{fig:wind}
\end{figure*}

\conclusions

In this paper, an automated AIR construction strategy is presented and compared with a traditional strategy
using data collected in Guttannen, Switzerland and Gangles, India.

The main purpose of this study was to quantify the influence of different fountain scheduling strategies on the
water use efficiency and volumes of AIRs exposed to identical weather conditions. We found that overwatering by
unscheduled fountains not just increased the fountain wastewater production but also enhanced the melting rate
of AIRs, mainly due to its surface albedo and fountain heat flux feedbacks. Scheduled fountains, in contrast,
consumed only 13 \% of the unscheduled fountain's water supply. However, the volume evolution of both the AIRs
showed no significant variations. 

Two different model forcings sensitive to the construction location's limited weather windows or water supply
were used to recommend two types of weather-sensitive discharge rates favouring higher volumes and better water
use efficiencies, respectively. Nevertheless, these models were able to capture more than 44 \% of the freezing
rate variations produced by the full energy balance model. Simulations comparing scheduled and unscheduled
fountains show that up to a two fold increase in water use efficiency is possible without compromising on
meltwater production.

The relationship of the maximum ice volume and water use efficiency to the size of the fountain water droplets
needs further examination. On one hand smaller droplets increase freezing rates due to their higher surface area
whereas on the other hand they also increase water losses since they are more prone to wind drift.

\appendix


\section{Automation system} \label{sec:auto_system}

The automation hardware consists of an AWS, flowmeter, control valve, drain valves, air valves, fountain,
pipeline and a logger. The logger feeds the AWS data to the automation software and informs the recommended
discharge rate to the flowmeter. The flowmeter adjusts the control valve to match the recommendation. In case a
termination criteria is valid, the drain and air valves allow the removal of water from the pipeline and entry
of air in the pipeline respectively.

The recommended discharge rate is equal to the ice mass change rate. However, certain termination criteria
override the discharge rate recommendation and drain the pipeline to prevent water loss or fountain freezing
events, namely: 

\begin{itemize}

\item High water loss is assumed if wind speed is greater than the user-defined critical wind speed.

\item High risk of fountain freezing event is assumed if $\frac{\Delta M_{ice}}{\Delta t}$ is lower than the user-defined minimum fountain discharge rate. 

\item Fountain freezing events are assumed if measured discharge rate is zero for at least 20 seconds and the pipeline is drained as a
  consequence.

\item Pipeline leakage is assumed if measured discharge rate is greater than the user-defined maximum fountain discharge rate.

\end{itemize}

\section{Model forcing based on water-use efficiency and maximum volume objectives} \label{sec:SEB}

The model complexity and data requirement \citep{balasubramanianInfluenceMeteorologicalConditions2022} were
reduced through assumptions that optimise for the ice volume or the water-use efficiency objectives. We define
the freezing rate and melting rate as the positive and negative mass change rate, respectively. Assumptions are
chosen, based on whether they overestimate/underestimate the freezing rate. HIV objective requires freezing rate
to be overestimated whereas water-use efficiency objective requires freezing rate to be underestimated. We
describe these two kinds of assumptions applied on each of the energy balance components below: 

\subsection{Surface Area $A_{cone}$ assumptions}

Determination of the surface area during the accumulation period is achieved by assuming a constant ice cone
radius equal to the fountain spray radius. The surface area scales the freezing rate of the AIR. Hence, for the
HIV objective, we assume the maximum possible slope of 1 for the ice cone or in other words $h_{cone} = r_{F}$.
Therefore, area is estimated as:  

\begin{equation} A_{cone} =\sqrt{2} \cdot \pi \cdot r_{F}^2  \end{equation}

Similarly, for the water-use efficiency objective, the area of the conical AIR is approximated to the area of
its circular base. Therefore, area is estimated as:

\begin{equation} A_{cone} =\pi \cdot r_{F}^2  \end{equation}

\subsection{Net shortwave radiation \texorpdfstring{$q_{SW}$}{Lg} assumptions}
\label{sec:SW}

The net shortwave radiation $q_{SW}$ is computed as follows:

\begin{equation} 
q_{SW} = (1- \alpha) \cdot ( SW_{direct} \cdot f_{cone} + SW_{diffuse})
\label{eqn:SW} 
\end{equation}

where $\alpha$ is the albedo value ; $SW_{direct}$ is the direct shortwave radiation; $SW_{diffuse}$ is the
diffuse shortwave radiation and $f_{cone}$ is the solar area fraction.

The data requirement was reduced by estimating the global shortwave radiation and pressure directly using the
location's coordinates and altitude through the solar radiation model described in
\citet{holmgrenPvlibPythonPython2018}. The algorithm used to estimate the clear-sky global radiation is
described in \citet{ineichenBroadbandSimplifiedVersion2008}.  

The diffuse and direct shortwave radiation is determined using the estimated global solar radiation as follows:

\begin{equation}
\begin{split}
  SW_{diffuse} &= cld \cdot SW_{global}\\
  SW_{direct} &= (1-cld) \cdot SW_{global}
\end{split}
\end{equation}

where $cld$ is the cloudiness factor. $cld$ is assumed to be 1 and 0 for the water-use efficiency and ice volume
objective respectively.

We ignore the variations in the albedo and assume it to be equal to snow albedo and ice albedo for the  ice
volume and water-use efficiency objective, respectively.

The solar area fraction $f_{cone}$ of the ice structure exposed to the direct shortwave radiation depends on the
shape considered. It is computed as

\begin{equation}
		f_{cone} =\frac{(0.5 \cdot r_{cone} \cdot h_{cone}) \cdot cos \theta_{sun} +(\pi \cdot
			{(r_{cone})}^2/2) \cdot sin \theta_{sun} }{\pi \cdot r_{cone} \cdot ({(r_{cone})}^2+{(h_{cone})}^2)^{1/2}}\\
\end{equation}

For the ice volume objective, since we assume the slope of the cone to be 1, $f_{cone}$ is determined as follows:

\begin{equation}
		f_{cone} =\frac{ cos \theta_{sun} + \pi \cdot sin \theta_{sun} }{2\sqrt{2} \cdot \pi }
\end{equation}

Similarly, for the water-use efficiency objective, since we assume the slope of the cone to be negligible, we get:

\begin{equation}
		f_{cone} =\frac{ sin \theta_{sun} }{2 }
\end{equation}

\subsection{Net Longwave radiation \texorpdfstring{$q_{LW}$}{Lg} assumptions} 

We assume $T_{ice} = 0 \degree C$ in order to determine outgoing longwave radiation. Since it is challenging to
constrain the minimum ice temperature, we maintain this assumption for both our objectives. However, in order to
estimate atmospheric emissivity, we again assume $cld$ to be 1 and 0 for the water-use efficiency and ice volume
objective respectively.

\subsection{Turbulent fluxes assumptions} \label{sec:Qs}

Turbulent fluxes estimation depend on the slope of the cone through the $\mu_{cone}$ parameter. As suggested 
by \citet{oerlemansBriefCommunicationGrowth2021}, we estimated this parameter as follows:

\begin{equation}
  \mu_{cone} =1 + s_{cone}/2
\end{equation}

Hence, the $\mu_{cone}$ parameter takes values of 1.5 and 1 for the ice volume and water-use efficiency
objective respectively.  Since turbulent fluxes impact both the freezing and the melting rates, this assumption
may not favor the corresponding objectives for certain sites.

\appendixtables   %% needs to be added in front of appendix tables

\begin{table}
  \caption{Free parameters in the model categorised as constant, model hyperparameters and weather 
  parameters with their respective values/ranges.}

	\label{tab:parameters}
	\begin{tabular}{lllll}
		\toprule

		\textbf{Constant Parameters}                       & \textbf{Symbol} & \textbf{Value} &
    \textbf{Unit} & \textbf{References} \\
    Van Karman constant & $\kappa$      & 0.4        &dimensionless & \citet{cuffeyPhysicsGlaciers2010}              \\
    Stefan Boltzmann constant & $\sigma$ & $\num{5.67 e-8} $& $W\, m^{-2}\, K^{-4}$ & \citet{cuffeyPhysicsGlaciers2010}\\
    Air pressure at sea level & $p_{0,a}$ & 1013 & $hPa$  & \citet{molgAblationAssociatedEnergy2004}\\
    Density of water & $\rho_{w}$ & 1000 & $kg\, m^{-3}$    & \citet{cuffeyPhysicsGlaciers2010}\\
    Density of ice & $\rho_{ice}$ & 917 & $kg\, m^{-3}$ & \citet{cuffeyPhysicsGlaciers2010}\\
    Density of air & $\rho_{a}$ &  1.29 & $kg\, m^{-3}$   & \citet{molgAblationAssociatedEnergy2004}\\
    Specific heat of water & $c_{w}$ & 4186 & $J\, kg^{-1}\,\degree C^{-1}$  & \citet{cuffeyPhysicsGlaciers2010}\\
    Specific heat of ice & $c_{ice}$ & 2097 & $J\, kg^{-1}\,\degree C^{-1}$ & \citet{cuffeyPhysicsGlaciers2010}\\
    Specific heat of air & $c_{a}$ & 1010 & $J\, kg^{-1}\,\degree C^{-1}$ & \citet{molgAblationAssociatedEnergy2004}\\
    Thermal conductivity of ice & $k_{ice}$ & 2.123  & $W\, m^{-1}\, K^{-1}$ & \citet{bonalesThermalConductivityIce2017} \\
    Latent Heat of Sublimation & $L_{s}$ & \num{2.848e6}  & $J\, kg^{-1}$ &   \citet{cuffeyPhysicsGlaciers2010}\\
    Latent Heat of Fusion & $L_{f}$ & \num{3.34e5} & $J\, kg^{-1}$ & \citet{cuffeyPhysicsGlaciers2010}\\
    Gravitational acceleration & $g$ & 9.81 & $m\, s^{-2}$ &\citet{cuffeyPhysicsGlaciers2010}\\
    Weather station height & $h_{AWS}$ & 2 & $m$ & assumed \\
    Model timestep                            & $\Delta t$            & $3600$           & $s$ & assumed \\\midrule

		\textbf{Model Hyperparameters} & \textbf{Symbol} & \textbf{Range} & \textbf{Unit} & \textbf{References} \\
    Surface layer thickness             & $\Delta x$            & $[\num{1e-2},\num{1e-1}]$           & $m$ & assumed
    \\\midrule
		\textbf{Weather Parameters} & \textbf{Symbol} & \textbf{Range} & \textbf{Unit} & \textbf{References} \\
    Ice Emissivity                      & $\epsilon_{ice}$      & $[0.95,0.99]$         & dimensionless & \citet{horiInsituMeasuredSpectral2006}             \\
    Surface Roughness                   & $z_0$                 & $[\num{1e-3},\num{5e-3}]$            & $m$  & \citet{brockMeasurementParameterizationAerodynamic2006}       \\
    Ice Albedo                          & $\alpha_{ice}$        & $[0.15,0.35]$         & dimensionless  &
    \citet{steinerModellingIcecliffBackwasting2015};            \\
    & &    &  & \citet{zollesRobustUncertaintyAssessment2019}      \\
    Snow Albedo                         & $\alpha_{snow}$       & $[0.8,0.9]$        & dimensionless  & \citet{zollesRobustUncertaintyAssessment2019}              \\
    Precipitation Temperature threshold & $T_{ppt}$             & $[0,2]$            & $\degree C$& \citet{shichangResponseZhadangGlacier2010}  \\
    Albedo Decay Rate                   & $\tau$                & $[10,22]$           & $days$ &
    \citet{schmidtImportanceAccurateGlacier2017};      \\
    & &    &  & \citet{oerlemansYearRecordGlobal1998}      \\\midrule
	\end{tabular}
\end{table}
\clearpage

\noappendix 

\bibliographystyle{copernicus}
\bibliography{zot_refs.bib}

\end{document}
