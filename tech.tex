\documentclass[utf8]{frontiersSCNS}
\usepackage{gensymb}
\usepackage{url,hyperref,lineno,microtype,subcaption}
\usepackage[onehalfspacing]{setspace}

\linenumbers
\usepackage{wasysym} % provides \DH, \dh, \Thorn, \thorn
% Leave a blank\usepackage{amsmath}
%\DeclareMathOperator{\sign}{sign} line between paragraphs instead of using \\

\usepackage{booktabs}
\usepackage{multirow}
\usepackage{siunitx} %for SI units

\def\keyFont{\fontsize{8}{11}\helveticabold }
\def\firstAuthorLast{Balasubramanian {et~al.}} %use et al only if is more than 1 author
\def\Authors{Suryanarayanan Balasubramanian\,$^{1}$, Martin Hoelzle\,$^{1}$}
\def\Address{$^{1}$University of Fribourg, Department of Geosciences, Fribourg, Switzerland\\} \def\corrAuthor{Suryanarayanan Balasubramanian}

\def\corrEmail{suryanarayanan.balasubramanian@unifr.ch}


\begin{document}
\onecolumn
\firstpage{1}

\title[Artificial Ice Reservoirs]{Optimal discharge rates for artificial ice reservoir (Icestupa) evolution}

\author[\firstAuthorLast ]{\Authors}
\address{}
\correspondance{}

\extraAuth{}

% \maketitle
\begin{abstract}
  Since 2014, mountain communities in Ladakh, India have been constructing dozens of Artificial Ice Reservoirs
  (AIRs) by spraying water through fountain systems every winter. The meltwater from these structures is crucial
  to meet irrigation water demands during spring. However, the water use efficiency (WUE) of this technology is
  poor due to the variability of AIR freezing rates due to the weather and fountain influences at the chosen
  location. This study compares the WUE between an AIR constructed manually with an AIR constructed using
  automated systems at Guttannen, Switzerland. The automation software uses a simplified equation with 6
  coefficients that capture the influence of temperature, humidity, wind and solar radiation variations on the
  freezing rate. Historical meteorological data in conjunction with the coordinates, altitude and time zone of
  the site are required to calculate these 6 coefficients. The automated AIR had a WUE three times more
  than the manual AIR. This is a promising result for dry mountain regions, where automated AIR technology could
  scale current mitigation efforts.

	\tiny
	\keyFont{ \section{Keywords:} icestupa, water storage, climate change adaptation, geoengineering } %All article types: you may provide up to 8 keywords; at least 5 are mandatory.
\end{abstract}

\section{Introduction}

\section{Study Sites and data}

\subsection{Meteorological data}

\subsection{Fountain observations}

\subsection{Drone surveys}

\section{Model setup}

\section{Results}

\subsection{Validation}

\section{Discussion}

\section{Conclusions}

\section{Appendix}

\section*{Conflict of Interest Statement} The authors declare that the research was conducted in the absence of
any commercial or financial relationships that could be construed as a potential conflict of interest.

\section*{Author Contributions} SB, MH, SW and FK designed the study.  SB developed the methodology with inputs
from MH.  MH, ML and JO reviewed the algorithm and helped improve it. SB processed the drone data. SB wrote the
model code. JB helped with model validation and uncertainty assessment. SB, MH, FK and SW participated in the
fieldwork.  SB led the writing of the paper and all co-authors contributed to it.

\section*{Funding} This work was supported and funded by the University of Fribourg and by the Swiss Government
Excellence Scholarship (SB). The associated fieldwork in India was supported by Himalayan Institute of
Alternatives and funded by the Swiss Polar Institute.

\section*{Acknowledgments} 
This work would not have been possible without the untiring efforts of the Swiss and
Indian icestupa construction teams through the winters of 2019, 2020 and 2021. We thank Mr. Adolf Kaeser and Mr.
Flavio Catillaz from Eispalast Schwarzsee (CH19); Daniel Beurki from the Guttannen Bewegt Association (CH20 and
CH21); Norboo Thinles, Nishant Tiku, Sourabh Maheshwari and the rest of the HIAL team (IN21).  We would also
like to thank Hansueli Gubler for designing the Swiss AWS; Dr. Tom Matthews for designing the Indian AWS;

\section*{Data Availability Statement} Model code is freely available in GitHub
(\url{https://github.com/Gayashiva/air_model}, last access: 17 December 2021) for non-profit purposes. The drone
data can be obtained from the authors upon request.

\bibliographystyle{frontiersinSCNS_ENG_HUMS} \bibliography{references}

\end{document}
