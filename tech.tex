\documentclass[tc, manuscript]{copernicus}

\usepackage{booktabs}
\usepackage{multirow}
\usepackage{siunitx} 

\begin{document}

\title{Fountain scheduling strategies for improving water-use efficiency of artificial ice reservoirs (Icestupas)}

\def\Authors{Suryanarayanan Balasubramanian\,$^{1,2}$, Martin Hoelzle\,$^{1}$Roger Waser\,$^{3}$}

\def\Address{$^{1}$University of Fribourg, Department of Geosciences, Fribourg, Switzerland $^{2}$University of
Applied Sciences and Arts, Luzern, Switzerland} \def\corrAuthor{Suryanarayanan Balasubramanian}
\Author[1,2]{Suryanarayanan}{Balasubramanian}
\Author[1]{Martin}{Hoelzle}
\Author[3]{Roger}{Waser}
\affil[1]{University of Fribourg, Department of Geosciences, Fribourg, Switzerland}
\affil[2]{Himalayan Institute of Alternatives, Ladakh, India}
\affil[3]{University of Applied Sciences and Arts, Luzern, Switzerland}

\correspondence{suryanarayanan.balasubramanian@unifr.ch}

\runningtitle{Scheduling AIR fountains}

\runningauthor{S. Balasubramanian}

\firstpage{1}

\maketitle

\begin{abstract}

  Artificial Ice Reservoirs (AIRs), often also called - Ice Stupas - are a climate change adaptation strategy
  developed in the Indian Himalayas (Ladakh). With this technology, otherwise unused stream/spring water is
  stored in large ice towers in winter. The surplus melt water that is generated in spring is used for
  satisfying irrigation water demands. Recent studies have shown that during construction of AIRs over 75\% of
  the water sprayed was lost. In this study, we examine whether fountain scheduling strategies can reduce this
  water loss by building two AIRs under identical weather conditions but with different fountain scheduling
  construction strategies. Fountain scheduling was realized through an automation system computing recommended
  discharge rates using real-time weather input and location metadata. Fountain operation using scheduling
  strategies produced similar ice volumes while consuming one-tenth of the water the unscheduled fountain used.
  Simulations converting unscheduled fountains to scheduled fountains improved the water use efficiency of
  several AIRs more than three fold. Overall, these results show that the automated fountain water supply
  management can increase the water use efficiency of AIRs and reduce their maintenance without compromising on
  their meltwater production.

\end{abstract}

\introduction

Cryosphere-fed irrigation networks in arid mountain regions are completely dependent on timely availability of
meltwater from snow, glaciers and permafrost \citep{immerzeelImportanceVulnerabilityWorld2020,
farhanHydrologicalRegimesConjunction2015, tveitenGlacierGrowingLocal2007}. With the accelerated decline of
glaciers due to climate change, these regions are experiencing seasonal water scarcity
\citep{hoelzleStatusRoleAlpine2019, xenariosAralSeaBasin2019, barandunStateFutureCryosphere2020}. This seasonal
water scarcity limits the output and duration of agricultural activities.

For example, in Ladakh, a cold arid desert in northern India, there is a typical shortage of water at the onset
of the agricultural season (April and May) until a sufficient and reliable supply of meltwater from glaciers
becomes available \citep{norphelSnowWaterHarvesting2015, nusserLocalKnowledgeGlobal2016,
vincentEnergyClimateChange2009}. 

To cope with this recurrent water scarcity, villagers have developed artificial ice reservoirs (AIRs) (see Fig.
\ref{fig:AIRforms} (a)). AIRs capture water in the autumn and winter, allowing it to freeze, and hold it until
spring, when it melts and flows down to irrigate the fields \citep{ipccChapterHighMountain2019,
vinceGlacierMan2009, clouseLadakhArtificialGlaciers2017, nusserSociohydrologyArtificialGlaciers2019}. In this
way, AIRs retain a previously unused portion of the annual flow and facilitate its use to compensate the
decreased flow during the following spring. 

A spirit of improvisation guides the construction strategy of AIRs making it difficult to classify them.
However, it has been found that construction strategies using fountain systems form AIRs which tend towards a
conical shape and those without form flat sheets of ice. Therefore, this study classifies all the AIRs produced
based on whether or not they use fountain systems. AIRs using fountain systems are called "ice stupas" (see Fig.
\ref{fig:AIRforms} (c)) and those without are called "ice terraces" (see Fig. \ref{fig:AIRforms} (b)) since this
terminology denotes the resulting shape of the respective AIRs appropriately. This study focuses on the ice
stupa form of AIRs.

\begin{figure}[htb]
\includegraphics[width=12cm]{Figures/AIR_forms.jpg}

\caption{(a) Schematic overview of the position of artificial ice reservoirs. These constructions are located at
altitudes between the glaciers and the irrigation networks in the cultivated areas. Ice terraces (b) and ice
stupas (c) are located at higher and lower altitudes respectively. Adapted from:
\cite{nusserLocalKnowledgeGlobal2016}}

\label{fig:AIRforms}
\end{figure}

Over the past decade, several ice stupas have been built to supplement irrigation water supply of mountain
villages in India \citep{wangchukIceStupaCompetition2020, palmerStoringFrozenWater2022,
aggarwalAdaptationClimateChange2021}, Kyrgyzstan \citep{bbcnewsBrightArtificialGlacier2020} and Chile
\citep{reutersConservationistsChileAim2021}. These AIRs are traditionally constructed by diverting springs or
glacial streams into fountain spray systems via embankments and pipelines. 

One of the common problems with AIR construction systems is related to fountain scheduling. Fountain scheduling
is simply answering the questions of “When do we spray?”, “How much do we spray?” and “How long do we spray?”.
Starting a fountain spray too early, spraying too much water or running a fountain spray too long might lead to
overwatering. At the very least, this practice wastes water.  Similarly, starting the fountain spray too late,
spraying too little water or not running the system long enough might lead to underwatering and can cause
reduced ice volumes.

Previous work \citep{balasubramanianInfluenceMeteorologicalConditions2022} has shown that traditional
construction systems suffer from overwatering. In order to avoid this issue, it is important to understand
surface freezing rates, which can be calculated by means of the full energy balance model developed in
\cite{balasubramanianInfluenceMeteorologicalConditions2022}. This model requires an accurate estimation of
fountain spray radius to produce recommended discharge rates. In theory, such an estimation is possible by
modelling the projectile motion of water droplets using fountain characteristics like aperture diameter, and
discharge rate. However, in practice, this estimation also depends on the relative importance of wind-driven
redistribution effects in determining the fountain spray radius. Therefore, estimating the fountain spray radius
requires a better understanding of the relative contribution of these two processes.

There are some practical issues that need to be addressed before dealing with the fountain scheduling processes.
For example, in the case of the Indian AIR, the fountain discharge rate could have been halved since they were
always two times higher than the modelled freezing rate
\citep{balasubramanianInfluenceMeteorologicalConditions2022}. However, in practice, reduction of discharge rate
could increase the maintenance cost due to higher risk of freezing events in the fountain pipeline.

An optimum construction strategy, therefore, should first prevent the occurrence of freezing events in the
fountain pipeline. These events can be prevented by setting a minimum threshold for the recommended discharge
rate. Additionally, discharge rates recommended need to be sensitive to constraints on the water supply or
weather of the construction site. For example, locations limited by their water supply like Ladakh, India
would prioritize water use efficiency whereas those limited by the duration of their favourable weather windows
like Guttannen, Switzerland would prioritize maximum ice volume. Accordingly, we use two types of model
parameter optimizations that prevent underwatering and overwatering to attain higher ice volumes and higher
water use efficiency respectively.

However, manually adjusting the fountain discharge rate is not practical due to two reasons- Firstly, this would
involve constant adjustments of discharge rates in response to the significant diurnal and seasonal variations
of the freezing rates. Secondly, frequent pipeline water drainage is required to avoid water losses. Therefore,
operation of scheduled fountains via automation systems is preferred to reduce the long-term maintenance costs.

The presented study was performed between two AIRs produced with and without automated fountain scheduling
strategies but exposed to identical meteorological conditions. The specific objectives of this study are to
compare the water-use efficiency, maximum ice volume and maintenance effort between these two construction
strategies.

\section{Study sites and data}

In this study, we used datasets presented in our previous work
\citep{balasubramanianInfluenceMeteorologicalConditions2022} along with new datasets. These old datasets record
the meteorological conditions and fountain characteristics of AIRs built in Gangles, India (IN21) and Guttannen,
Switzerland (CH21) during the winter of 2020-21. In this section, we focus on describing the new AIR datasets
collected in Guttannen, Switzerland during the winter of 2021-22 (CH22).

The Guttannen site (46.66 $\degree$N, 8.29 $\degree$E) is situated in the Berne region, Switzerland and has an
altitude of 1047 $m$ a.s.l. In the winter (Oct-Apr), mean daily minimum and maximum air temperatures vary
between -13 and 15 $\degree C$. Clear skies are rare, averaging around 7 days during winter. Daily winter
precipitation can sometimes be as high as 100 $mm$. These values are based on 30 years of hourly historical
weather data measurements \citep{meteoblueClimateGuttannen2021}. Two AIRs were constructed by the Guttannen
Bewegt Association, the University of Fribourg and the Lucerne University of Applied Sciences and Arts during
the winters of 2021-22 using a traditional and an automated construction strategy.

\begin{figure}[htb]
\includegraphics[width=12cm]{Figures/AIR_fountains.jpg}
\caption{Unscheduled and scheduled fountains used for construction of traditional and automated AIRs at Guttannen. Picture credits: Daniel Bürki}
\label{fig:2AIR}
\end{figure}

The automated and the traditional AIRs were constructed adjacent to each other but with different fountain
designs as shown in Fig. \ref{fig:2AIR}. This ensured both AIRs shared the same water source and identical
weather conditions. In addition, a webcam guaranteed a continuous surveillance of the automated AIR.   

In the traditional construction strategy, tree branches were laid covering the fountain pipe to initiate and
speed up the ice cone formation process. In the automated strategy, only the fountain pipe was placed before the
water spray started. The construction of both the AIRs began on 8th December 2021 on a snow bed of 13 cm thickness
and ended on 12th April 2022. These two dates are referred as start and expiry dates henceforth.

In the traditional strategy, the fountain was operated manually whereas in the automated strategy the fountain
discharge rate was controlled using real time weather input and several control parameters which could be
modified via a user interface. Henceforth, we refer to the fountain used in the traditional and automated
construction strategy as unscheduled and scheduled fountains, respectively.


\subsection{Meteorological data}

Air temperature, relative humidity, wind speed, pressure, precipitation, incoming long-wave radiation,
short-wave radiation and cloudiness index are required to calculate the surface energy balance of an AIR. The
primary weather data source was an automatic weather station (AWS) located around 20 m away. Hourly ground
temperature measurements were also recorded by the AWS to approximate the fountain's water temperature. Less
than 0.4 \% of the data was found to be missing and the data gaps were filled by linear interpolation. However,
two additional datasets were required to obtain all the necessary input variables, namely, cloudiness index and
precipitation. These two datasets were obtained from ERA5 reanalysis dataset
\citep{hersbachERA5GlobalReanalysis2020} and a MeteoSwiss AWS located 184 m away (Station ID: 0-0756-0-GTT)
respectively.

\begin{figure*}[htb]
\includegraphics[width=12cm]{Figures/disvstemp.png}
\caption{Temperature and discharge measurements at the Guttannen construction site}
\label{fig:aws} 
\end{figure*}

\subsection{Fountain observations}

The scheduled and unscheduled fountains have four attributes, namely: discharge rate ($Q$), height ($h$), water
temperature ($T_F$), nozzle pressure loss ($P_{nozzle}$), and the spray radius ($r$). Discharge rate represents
the discharge rate of the water in the fountain pipeline. Height denotes the height of the fountain pipeline
installed. Fountain water temperature is the temperature of water droplets produced by the fountain. The nozzle
pressure loss denotes the preesure consumed during the formation of water droplets. Spray radius denotes the
observed ice radius formed from the fountain water droplets.

The height was increased in steps of 1 meter for both fountains. For the scheduled fountain, the
construction began with a height of 3 $m$ with an increase to 4 $m$ on 23rd December. For the unscheduled
fountain, the construction began with a height of 3.7 $m$ and was increased two times on 23rd December
and 12th February.

Fig. \ref{fig:aws} shows the temporal variation in temperature and discharge rate of both the scheduled and
unscheduled fountains. The unscheduled fountain had variations in discharge rate whenever the fountain height
was increased. The scheduled fountain's discharge rate variations were caused by the control valve of the
automation system. The control ball valve position between 0 to 100 \% (or 0 to 90 $\degree$) was regulated
based on real-time meteorological conditions. Throughout the study period, the control valve was opened
completely (100 \%) only once corresponding to the time when the temperature attained its minimum of -13
$\degree \, C$ on 20th December. After this event, the control valve was never opened beyond 34 \%.  

The unscheduled fountain was manually operated to spray all the available discharge until a fountain freezing
event interrupted the discharge on 17th February. Unfortunately, no discharge rate measurements were recorded
for the unscheduled fountain. However, the unscheduled fountain was observed to have a higher discharge rate
compared to the scheduled fountain due to its higher aperture area (see Fig. \ref{fig:2AIR}). Therefore, we
conservatively assume the discharge rate of the unscheduled fountain to be equal to the maximum discharge rate
of the scheduled fountain which was observed to be 13 $l/min$ and 11 $l/min$ at a fountain height of 3 $m$ and 4
$m$ respectively.

The water temperature of both the fountains were estimated from the AWS ground temperature dataset. This dataset
was acquired through a thermistor located 0.3 $m$ below the base of the fountain.

\subsection{Drone surveys}

Several photogrammetric surveys were conducted on the traditional and the automated AIRs. The details of these
surveys and the methodology used to produce the corresponding outputs are explained in
\cite{balasubramanianInfluenceMeteorologicalConditions2022}. The digital elevation models (DEMs) generated from
the obtained imagery were analysed to document the ice radius, the surface area and the volume of the ice
structures. Ice radius measurements of drone flights which observed either an increase in AIR circumference or
volume were averaged to determine the fountain's spray radius. The number of drone surveys conducted for the
traditional and the automated AIRs were 8 $m$ and 6 $m$, respectively (see Table \ref{tab:uav}). 

\begin{table}
	\centering
	\caption{Summary of the drone surveys}
	\label{tab:uav}
	\begin{tabular}{@{}|llllll|@{}}
		\toprule
		\textbf{}              & \textbf{No.} & \textbf{Date} & \textbf{Volume} & \textbf{Radius} & \textbf{Surface Area} \\ \midrule
		\multicolumn{1}{|l|}{\multirow{8}{*}{\rotatebox[origin=c]{90}{Traditional}}}
		                       & 1            & Dec 23, 2021  & 17 $m^{3}$     & 2.9 $m$
		                       & 47 $m^{2}$                                                                      \\
		\multicolumn{1}{|l|}{} & 2            & Jan 3, 2022  & 22 $m^{3}$     & 3.4 $m$
		                       & 61 $m^{2}$                                                                      \\
		\multicolumn{1}{|l|}{} & 3            & Jan 22, 2022   & 35 $m^{3}$     & 4 $m$
		                       & 79 $m^{2}$                                                                      \\
		\multicolumn{1}{|l|}{} & 4            & Feb 6, 2022  & 44 $m^{3}$     & 4.2 $m$
		                       & 86 $m^{2}$                                                                      \\
		\multicolumn{1}{|l|}{} & 5            & Feb 20, 2022  & 43 $m^{3}$     & 4.3 $m$
		                       & 86 $m^{2}$                                                                      \\
		\multicolumn{1}{|l|}{} & 6            & Mar 19, 2022  & 33 $m^{3}$     & 4.4 $m$
		                       & 84 $m^{2}$                                                                      \\
		\multicolumn{1}{|l|}{} & 7            & Mar 26, 2022  & 24 $m^{3}$     & 4.3 $m$
		                       & 74 $m^{2}$                                                                      \\
		\multicolumn{1}{|l|}{} & 8            & Apr 12, 2022  & 11 $m^{3}$     & 3.5 $m$
		                       & 50 $m^{2}$                                                                      
		\\\midrule
		\multicolumn{1}{|l|}{\multirow{6}{*}{\rotatebox[origin=c]{90}{Automated}}}
		                       & 1            & Dec 23, 2021  & 35 $m^{3}$      & 4.3 $m$
		                       & 73 $m^{2}$                                                                       \\
		\multicolumn{1}{|l|}{} & 2            & Jan 3, 2022   & 32 $m^{3}$      & 4.4 $m$
		                       & 81 $m^{2}$                                                                       \\
		\multicolumn{1}{|l|}{} & 3            & Feb 20, 2022   & 60 $m^{3}$      & 5.3 $m$
		                       & 105 $m^{2}$                                                                       \\
		\multicolumn{1}{|l|}{} & 4            & Mar 19, 2022   & 28 $m^{3}$      & 3.7 $m$
		                       & 57 $m^{2}$                                                                       \\
		\multicolumn{1}{|l|}{} & 5            & Mar 26, 2022   & 19 $m^{3}$      & 3.7 $m$
		                       & 53 $m^{2}$                                                                       \\
		\multicolumn{1}{|l|}{} & 6            & Apr 12, 2022   & 7 $m^{3}$      & 2.5 $m$
		                       & 53 $m^{2}$                                                                       \\
		\bottomrule
	\end{tabular}

\end{table}

\section{Methods}

\subsection{Discharge scheduling software}

Recommended discharge rates can only be produced if more information about the AIR surface properties and
weather conditions are available. Particularly, resolving the uncertainty in the expected freezing rate requires
quantification of the following three model variables: slope, albedo and cloudiness. But these properties cannot
be predicted beforehand. Therefore, we instead associate the upper and lower bound of each variable to a
different model depending on whether they increase the freezing rate or not. Higher albedo values decrease the shortwave
radiation impact. Higher cloudiness values increase both the shortwave and the longwave radiation impact. The
model overestimating the freezing rate will be referred to as ice volume optimised model (IVOM) and the model
underestimating the freezing rate will be referred to as water-use efficiency optimised model (WEOM),
respectively. Accordingly, the values assigned for all the three variables in the respective model is shown in
Table \ref{tab:assumptions}.

The discharge scheduling software implements two types of fountain scheduling strategies depending on which
model type is suitable. WEOM model type is used if the location has limited quantity since it is expected to
produce better water-use efficiency. IVOM model type is used if the location had limited duration of favourable
weather windows since it is expected to produce higher ice volumes. These two kinds of scheduled fountains will
be referred to as water-sensitive fountain and weather-sensitive fountain henceforth.

\begin{table}[htb]
\centering
\caption{Assumptions for the parametrisation introduced to simplify the ice volume optimised model (IVOM) and
water-use efficiency optimised model (WEOM).}
\label{tab:assumptions}
\begin{tabular}{@{}lllll@{}}
\toprule
\textbf{Estimation of} & \textbf{Symbol} & \textbf{IVOM} & \textbf{WEOM} & \\ \midrule
\multicolumn{1}{|l}{Slope}        & $s_{cone}$ & $ 1 $ & $0$ & \multicolumn{1}{l|}{} \\ \midrule
\multicolumn{1}{|l}{Albedo} & $\alpha$ & $\alpha_{snow}$ & $\alpha_{ice}$ & \multicolumn{1}{l|}{} \\\midrule 
\multicolumn{1}{|l}{Cloudiness}  & $cld$ & $0$ & $1$ & \multicolumn{1}{l|}{} \\ \bottomrule
\end{tabular}
\end{table}

We apply the assumptions described in Table \ref{tab:assumptions} on the one-dimensional description of energy
fluxes as used in \cite{balasubramanianInfluenceMeteorologicalConditions2022} to obtain the rate of change of
AIR ice mass as follows: 

\begin{equation}
  \frac{\Delta M_{ice}}{\Delta t}  =  (\frac{q_{SW} + q_{LW} + q_{S} + q_{F} + q_{R} + q_{G} - q_{T}}{L_F} + \frac{q_{L}}{L_V} ) \cdot A_{cone}
	\label{eqn:auto}
\end{equation}

Upward and downward fluxes relative to the ice surface are positive and negative, respectively. The first term
represents the mass change rate due to freezing of the fountain water and melting of the ice. $q_{SW}$ is the
net short-wave radiation; $q_{LW}$ is the net long-wave radiation; $q_{L}$ and $q_{S}$ are the turbulent latent
and sensible heat fluxes; $q_{F}$ is the fountain discharge heat flux; $q_{R}$ is the rain water heat flux;
$q_{G}$ is the ground heat flux; $q_{T}$ is the temperature heat flux and $A_{cone}$ is the area of the AIR
surface. The derivation of these individual terms for the IVOM and WEOM model versions are discussed in 
Appendix \ref{sec:SEB}.

Equation \ref{eqn:auto} is implemented in the automation software. The user interface of the software enables
input of the spray radius, altitude, latitude and longitude of the construction location. The automation
hardware consists of an AWS, flowmeter, control valve, drain valves, air valves, fountain, pipeline and a
logger. The logger feeds the AWS data to the automation software and informs the recommended discharge rate to
the flowmeter. The flowmeter adjusts the control valve to match the recommendation. In case a termination
criteria gets met, the drain and air valves begin to allow the removal of water from the pipeline and entry of
air in the pipeline respectively.

The recommended discharge rate is equal to the mass change rate. However, certain termination criteria listed
below override the discharge rate recommendation and drain the pipeline to prevent water loss or fountain
freezing events:

\begin{itemize}

\item High water loss is assumed if wind speed is greater than the user-defined critical wind speed.

\item High risk of fountain freezing event is assumed if mass change rate is lower than the user-defined minimum fountain discharge rate. 

\item Freezing events in the fountain pipeline are assumed if measured discharge rate is zero for at least 20
  seconds. 

\item Pipeline leakage is assumed if measured discharge rate is greater than the user-defined maximum fountain discharge rate.

\end{itemize}

\section{Modelling fountain spray radius} \label{sec:r_F}

The fountain spray radius is defined as the largest horizontal distance covered by fountain water droplets. This
can be determined by modelling the trajectory of these droplets using the projectile motion equation. This
projectile motion starts at the fountain nozzle and ends at the AIR surface.  To obtain the droplet speeds
($v$), we use the the measured aperture diameter ($dia = 0.001 m$) and discharge rate of the scheduled fountain
through the following equation:

\begin{equation}
	\label{eqn:dis}
v = \frac{4 \cdot Q}{60 \cdot 1000 \cdot \pi \cdot dia^2}
\end{equation}

where $v$ is the droplet speed in $m/s$ and $Q$ is the discharge rate of the fountain in $l/min$

To obtain the spray radius ($r$), we use the optimum launch angle $\theta = 45 \degree$ in the projectile motion
equation to get:

\begin{equation}
  \label{eqn:radf}
  r = \frac{v \cdot(v + \sqrt{v^2 + 4hg)}}{2g}
\end{equation}

The influence of wind-driven redistribution can be included in the spray radius by multiplying the wind speed
with the time of flight of the water droplets.

\section{Determination of pressure losses} \label{sec:p_loss}

The fountain pipeline system delivering water to the ice stupa suffers from several pressure losses. These
losses limit the maximum height that the fountain can achieve. There are three kinds of losses namely, (a)
altitudinal ($P_{alt}$), (b) frictional ($P_{friction}$) and (c) nozzle ($P_{nozzle}$) losses. The altitudinal
losses depend on the altitude difference between the source and the fountain. The frictional losses are
proportional to the length of the pipeline and inversely proportional to their diameter. The nozzle losses
depend on the engineering design of the fountain nozzle.

The pressure losses can be determined using the Bernoulli equation as follows:

\begin{equation}
  \label{eqn:pressure}
  P_{source} = P_{alt} + P_{friction} + P_{nozzle} + \frac{\rho \cdot v^2}{2} \cdot 10 ^{-5}
\end{equation}

where $P_{source}$ is the source pressure , $P_{nozzle}$ is the pressure loss due to the fountain nozzle and
$P_{alt}$ is pressure loss due to the altitudinal difference between the pipeline input and fountain output. All
these pressure variables are measured in bars.  The velocity $v$ can be determined from discharge rate
observations using Eqn. \ref{eqn:dis}. 

The frictional loss of the pipeline used in the experiment can be determined from the well known
Hagen–Poiseuille equation \cite{poiseuilleExperimentalInvestigationsFlow1847}:  

\begin{equation}
  \label{eqn:friction}
  P_{friction} = \frac{3.2 \cdot \mu \cdot v \cdot L}{\rho \cdot g \cdot dia^2}
\end{equation}

where $P_{friction}$ is in bars, $L$ is the total length of the pipeline measured in meters and $v$ is the water velocity in
$m/s$. Note that the above equation only applies for laminar flow which was the situation in our case.


\subsection{Model updates}

In this article, we refrain from a more general model description and focus only on the description of the
integration of fountain scheduling processes with the AIR model
\citep{balasubramanianInfluenceMeteorologicalConditions2022}. For details on the model internals and the
calculation of surface processes we refer to the respective literature references. 

In the previous version of the model \citep{balasubramanianInfluenceMeteorologicalConditions2022}, the fountain
water temperature ($T_F$) was estimated as a constant parameter. However, in reality, this is a poor
approximation as it is not accounting for two processes, namely, (a) temperature fluctuations during transit
from the source to the fountain nozzle; (b) temperature fluctuations during the flight time of the water
droplets after leaving the fountain nozzle. Therefore, we instead use measured hourly ground temperature
values to approximate process (a) and assume water temperature cools down to 0 $\degree\,C$ during subzero
air temperature conditions to approximate process (b).

In the previous version of the model \citep{balasubramanianInfluenceMeteorologicalConditions2022}, fountain
discharge events were reset from surface albedo to ice albedo. However, this assumption limits the accuracy of
the model, especially, for the automated AIR, where several fountain discharge events of short duration occur.
Therefore, we assumed that discharge events instead reduce the albedo decay rate ($\tau$) by a 
factor of $\frac{\alpha_{ice}}{\alpha_{snow}}$.

Additionally, both the AIRs experienced many precipitation events. Therefore, it was no longer accurate to
assume AIR density ($\rho_{cone}$) to be equal to ice density. We instead parameterised AIR density $\rho_{cone}$ as follows:

\begin{equation}
  \rho_{cone} = \frac{M_{F} + M_{dep} + M_{ppt}}{(M_{F} + M_{dep})/\rho_{ice} + M_{ppt}/\rho_{snow}}
\end{equation}

where $M_F$ is the cumulative mass of the fountain discharge; $M_{ppt}$ is the cumulative precipitation;
$M_{dep}$ is the cumulative accumulation through water vapour deposition; $\rho_{ice}$ is the ice density (917
$kg\,m^{-3}$) and $\rho_{snow}$ is the density of wet snow (300 $kg\,m^{-3}$) taken from
\cite{cuffeyPhysicsGlaciers2010} .

Rain events were not considered in the previous version of the model but they occurred in our experiment. The
influence of rain events on the albedo and the energy balance was assumed to be similar to discharge events.
However, the water temperature of a rain event was assumed to be equal to the air temperature. Accordingly, the
rain water heat flux ($q_{R}$) generated due to a rain event was equal to:

\begin{equation}
  q_{R} = \frac{\Delta M_{ppt} \cdot c_{water} \cdot T_{a}}{\Delta t \cdot A_{cone}}
\end{equation}

where $M_{ppt}$ is the hourly precipitation in meters, $c_{water}$ is the specific heat of water, and $A_{cone}$
is the surface area.

\subsection{Calibration}

The model parameters were calibrated to the median values of the ranges presented in Appendix Table
\ref{tab:parameters}. However, the surface layer thickness parameter was calibrated to a value of 0.09 $m$ for the
automated AIR instead of the default value of 0.05 $m$. This calibration was necessary to prevent hourly surface
temperature fluctuations to assume unphysical values above 40 $\degree\,C$.

We performed the validation of the model on the traditional and automated AIRs by evaluating the root mean
squared error (RMSE) between volume estimates and measurements. 

The performance of the IVOM and WEOM versions of the physical model were assessed by comparing correlation of its
discharge rate estimates with the validated freezing rate of the traditional AIR.

\section{Results}

\subsection{Model validation}

The volume estimation for the automated and traditional AIR had a RMSE of 8 $m^3$ and 6 $m^3$ with the drone
volume observations, respectively. These are within 13 \% and 11 \% of the maximum volume of the automated and
the traditional AIR respectively. The estimated and measured AIR volumes are shown in Fig. \ref{fig:validation}.  

\begin{figure*}[t] \includegraphics[width=12cm] {Figures/validation.png} \caption{Volume validation of the
scheduled and unscheduled fountain construction strategies.} \label{fig:validation} \end{figure*}

\begin{figure*}[t]
\includegraphics[width=12cm]{Figures/dis_processes.png}
\caption{(a) Surface albedo  and (b) fountain discharge heat flux showed significant variations between the two
  AIRS due to the differences in their discharge rates.}
\label{fig:dis_processes}
\end{figure*}

\begin{table}
	\centering
	\caption{Summary of the mass balance, energy balance, fountain and AIR characteristics estimated at the end of the respective
  simulation duration for the automated and the traditional AIRs}
	\label{tab:mb}
	\begin{tabular}{@{}|llllll|@{}}
		\toprule
		\textbf{}              & \textbf{Name}                   & \textbf{Symbol} & \textbf{Traditional} & \textbf{Automated} &
		\textbf{Units}                                                                                                       \\ \midrule
		\multicolumn{1}{|l|}{\multirow{3}{*}{\rotatebox[origin=c]{90}{Input}}}
		                       & Fountain discharge              & $M_F$           & \num{1.1e6}   & \num{1.5e5}     & $kg$  \\
		\multicolumn{1}{|l|}{} & Snowfall                        & $M_{ppt}$       & \num{9.2e3}   & \num{1.4e4}   & $kg$  \\
		\multicolumn{1}{|l|}{} & Deposition                      & $M_{dep}$       & \num{4.0e2}   & \num{4.5e2}     & $kg$  \\ \midrule
		\multicolumn{1}{|l|}{\multirow{4}{*}{\rotatebox[origin=c]{90}{Output}}}
		                       & Meltwater                       & $M_{water}$     & \num{4.5e4} & \num{5.4e4}   & $kg$  \\
		\multicolumn{1}{|l|}{} & Ice                             & $M_{ice}$       & \num{7.4e3} & \num{6.1e3}    & $kg$  \\
		\multicolumn{1}{|l|}{} & Sublimation                     & $M_{sub}$       & \num{3.7e3} & \num{4.5e3}     & $kg$  \\
		\multicolumn{1}{|l|}{} & Fountain wastewater             & $M_{waste}$     & \num{1.07e6} & \num{1.0e5}     & $kg$  \\ \midrule
		\multicolumn{1}{|l|}{\multirow{7}{*}{\rotatebox[origin=c]{90}{Energy Flux}}}
                           & Shortwave radiation             &  $q_{SW}$       & $14$  & $21$ & \% \\
		\multicolumn{1}{|l|}{} & Longwave radiation              &  $q_{LW}$       & $25$  & $25$ & \% \\
		\multicolumn{1}{|l|}{} & Sensible heat                   &  $q_{S}$        & $38$   & $33$ & \% \\
		\multicolumn{1}{|l|}{} & Latent heat                     &  $q_{L}$        & $19$  & $19$ & \% \\
		\multicolumn{1}{|l|}{} & Fountain discharge heat         &  $q_{F}$        & $4$  & $0$     & \% \\
		\multicolumn{1}{|l|}{} & Rain heat                       &  $q_{R}$        & $0$  & $0$     & \% \\
		\multicolumn{1}{|l|}{} & Ground heat                     &  $q_{G}$        & $1$   & $1$     & \% \\\midrule
		\multicolumn{1}{|l|}{\multirow{2}{*}{\rotatebox[origin=c]{90}{AIR}}}

		                       & Maximum AIR Volume              &                 & 53            & 61            & $m^{3}$ \\
		\multicolumn{1}{|l|}{} & Water Use Efficiency            &                 & 4             & 35            & \% \\\midrule
	\end{tabular}
\end{table}

\subsection{Comparison of AIR construction strategies}

Table \ref{tab:mb} shows how the two different fountain scheduling strategies influence the mass and energy balance
of the respective AIR.  The overall impact of the radiation fluxes (long-wave and short-wave) and the
turbulent fluxes (sensible and latent) on the freezing and melting energies is determined from their 
energy turnover. The energy turnover is calculated as the sum of energy fluxes in absolute values (see Table
\ref{tab:mb}). 

Fountain scheduling reduced the fountain discharge input and fountain wastewater output by an order of
magnitude. However, this is not resulting in an appreciable difference in the volume evolution of the automated
and traditional AIRs as shown in Fig. \ref{fig:simvsreal}. This is due to two counteracting surface processes
during fountain spray: (a) dampening of albedo to ice albedo and (b) absorption of the heat energy of the
fountain water droplets. The temporal variation of the magnitude of these processes are shown in Fig.
\ref{fig:dis_processes}.

There is a considerable difference in the contribution of the shortwave radiation due to the effect of process
(a). Even though the unscheduled fountain was active for a much longer duration, the frequent snowfall events
counteracted the albedo feedback of its fountain discharge. In contrast, the albedo of the automated AIR was
reduced by late fountain spray events particularly in the months of March and April as shown in Fig.
\ref{fig:dis_processes}. These poorly timed fountain spray events occurred because the global solar radiation
diurnal variation of the automation system is calibrated based on values for the month of February. Therefore,
poor calibration of the automation system resulted in an increased impact of shortwave radiation on the
automated AIR. Similarly, the fountain discharge heat flux for the traditional AIR was enhanced due to process
(b). The higher discharge quantity of the unscheduled fountains and its longer duration were responsible for the
higher contribution of fountain discharge heat flux in the overall energy turnover. Therefore, higher melt of
the automated AIR due to process (a) counteracted the higher melt of the traditional AIR due to process (b).

\subsection{Performance of weather and water sensitive fountains}

The water-sensitive and weather-sensitive fountains scheduled by the WEOM and IVOM model versions estimated the
freezing rate of the unscheduled fountain with a correlation of 0.4 and a RMSE less than 0.8 $l/min$ and 1.8
$l/min$, respectively. The discharge rate values of the weather-sensitive fountain overestimated the freezing
rate 93 \% of the fountain spray duration whereas those of the water-sensitive fountain overestimated the
freezing rate 70 \% of the unscheduled fountain spray duration as illustrated by Fig. \ref{fig:simvsreal}.
Therefore, the IVOM model version was successful in prioritizing the maximum ice volume by overestimating the
discharge rates but the WEOM model version could not underestimate its discharge rate values sufficiently to
optimize for water use efficiency.

\begin{figure*}[htb]
\includegraphics[width=8cm]{Figures/simvsreal.jpg}

\caption{ Comparison of the freezing rate estimated for the unscheduled fountain and the discharge rate of the
scheduled fountains. }

\label{fig:simvsreal}
\end{figure*}

\subsection{Benefits of scheduling fountains}

The difference in water-use efficiency and maximum ice volume between unscheduled and scheduled fountains in the two
locations across two winters are presented in Fig. \ref{fig:wue} (a). Four experimental values (highlighted by
circles) are shown together with five simulated values (highlighted by squares).  The experimental values were
taken from the IN21 and CH21 AIRs studied in \citet{balasubramanianInfluenceMeteorologicalConditions2022} and
the CH22 AIR presented in this study. 

\begin{figure*}[htb]
\includegraphics[width=\textwidth]{Figures/wue.png}

\caption{(a) The maximum volumes and water-use efficiency estimated for AIRs constructed in different locations
(represented by colours) with different fountain scheduling strategies (represented by symbols). Experimental
values are highlighted by circles and simulated values are highlighted by squares. (b) Comparison of
the unscheduled and scheduled fountain's discharge rates at the IN21 location.}

\label{fig:wue}
\end{figure*}

The water-use efficiency of all the unscheduled fountains are below 20 \%. In general, the water-use efficiency
increases more than three folds when the weather-sensitive or water-sensitive fountains are used in both
locations.  

For the Indian location, the three kinds of fountains yield significantly different results.  The discharge
duration and the max discharge rate of the three IN21 fountains were responsible for these different results
(see Fig. \ref{fig:wue} (b)). The max discharge rate of the unscheduled fountain was more than twice that of
scheduled fountains resulting in a higher water loss. Freezing events in the fountain pipeline caused frequent
interruptions in the unscheduled discharge rate (see Fig. \ref{fig:wue} (b)). In contrast, the mean freezing
rates of the other two fountains during these events were above their median values. This is because, very cold
temperatures freeze the water inside rather than outside the fountain system instigating these freezing events in
the fountain pipeline. Therefore, both the discharge duration and the mean freezing rate of the unscheduled
fountain was much lower resulting in lower ice volumes. The water-sensitive fountain underestimated the freezing
rate during the construction period and therefore produced much lower ice volumes compared to the
weather-sensitive fountain. 

For the Swiss locations, scheduled fountains yielded better water-use efficiency but did not alter the maximum
volume obtained significantly. 


\subsection{Pressure losses}

\begin{table}[htb]
\centering
\caption{Pipeline configuration of the automated icestupa.}
\label{tab:pipe}
\begin{tabular}{@{}llll@{}}
\toprule
\textbf{Name} & \textbf{Symbol} & \textbf{Value} & \\ \midrule
\multicolumn{1}{|l}{Pipeline diameter}      & $dia$ & 16 $mm$ & \multicolumn{1}{l|}{} \\ \midrule
\multicolumn{1}{|l}{Pipeline length}        & $L$ & 66 $m$ & \multicolumn{1}{l|}{} \\ \midrule
\multicolumn{1}{|l}{Source water pressure} & $P_{source}$ & 6 $bar$  & \multicolumn{1}{l|}{} \\\midrule 
\multicolumn{1}{|l}{Altitudinal pressure head}  & $P_{alt}$ & 1.1 $bar$ & \multicolumn{1}{l|}{} \\ \midrule
\multicolumn{1}{|l}{Water viscosity}  & $\mu$ & 0.00152 $Pa\,s$ & \multicolumn{1}{l|}{} \\ \bottomrule
\end{tabular}
\end{table}

The pressure consumption across the fountain pipeline provide insights into how the fountain pipeline
configuration can be better optimised. The pipeline configuration of the automated icestupa fountain is listed
in Table \ref{tab:pipe}. Maximum frictional loss occurs during maximum discharge which was measured to be 11
$l/min$. Substituting the corresponding values in Eqn. \ref{eqn:friction}, we get $P_{friction}$ to be 0.3
$bar$. The velocity $v$ can be determined from our discharge rate observation from Eqn. \ref{eqn:radf}.
Therefore, from Eqn. \ref{eqn:pressure}, we get $P_{nozzle}$ to be 4.6 $bar$ which is more than 75 \% of the
source water pressure.

\subsection{Influence of wind driven redistribution on fountain spray radius}

The estimated volume changes over the month of January of the Swiss AIRs built in the winter of 2021-22 is less
than half that of AIRs from the previous winter (CH21). One would expect this difference is due to warmer
temperatures during the CH22 winter. However, the median January temperature of CH22 winter was colder than the
CH21 winter (see Fig. \ref{fig:CH_diffs} (a)). Moreover, the volume growth of CH20 AIR is 6 folds of CH22 AIR,
despite CH20 winter being 3 $\degree C$ warmer.

We suspect the primary driver of volume difference across different winters to be the spray radius (see
Fig. \ref{fig:CH_diffs} (b)). However, this observation contradicts our expectation that AIRs using the same
water source and fountain designs have similar spray radius. Moreover, manual measurements of the fountain spray
radius were observed to be lesser than the drone observations of the ice radius. These two observations imply
that wind drift of water droplets could play a major role in temporal fluctuations of the ice radius.

\begin{figure*}[htb]
\includegraphics[width=\textwidth]{Figures/CH_diffs.jpg}

\caption{(a) Estimated volume change and median temperature (b) Observed spray radius and median wind speed
during January for AIRs built across three winters. } 

\label{fig:CH_diffs} 
\end{figure*}

To validate this hypothesis, we modelled the projectile motion of scheduled fountain water droplets with wind speed
values taken from CH22 and CH21 experiments, respectively. Fig. \ref{fig:wind} shows the modelled spray radius
produced using these two wind datasets and compares them with the measured spray radius values. As illustrated,
wind speed drives the temporal variation in the spray radius. Moreover, the spray radius of the scheduled
fountain with CH22 wind values is much higher than when using CH21 wind values. Therefore, the determination of
the fountain spray radius cannot be performed using the characteristics of the fountain nozzle alone since it is
significantly influenced due to the temporal variation of the wind speed.

\begin{figure*}[htb]
\includegraphics[width=12 cm]{Figures/radf.png}
\caption{Modelled spray radius using wind values from CH22 and CH21 experiments. Measured spray radius are
indicated as dots.}
\label{fig:wind}
\end{figure*}

\section{Discussion}


\subsection{Additional water losses}

In practice, parts of the water volumes exiting the fountains do not reach the ground due to both thermodynamic
(evaporation and sublimation) and mechanical (wind-driven redistribution) effects. While these water losses can
be significant \citep{hanzerSimulationSnowManagement2020}, simulating them using physical formulation is
challenging since it is sensitive to the diameter of water droplets produced by the fountain.

\conclusions

In this paper, an automated AIR construction strategy is presented and compared with a traditional strategy
using data collected in Guttannen, Switzerland and Gangles, India.

The main purpose of this study was to quantify the influence of different fountain scheduling strategies on the
water use efficiency and ice volumes of AIRs exposed to identical weather conditions. We found that overwatering by
unscheduled fountains not just increased the fountain wastewater production but also enhanced the melting rate
of AIRs, mainly due to its surface albedo and fountain heat flux feedbacks. Scheduled fountains, in contrast,
consumed only 13 \% of the unscheduled fountain's water supply. However, the volume evolution of both the AIRs
showed no significant variations. 

Two different model forcings sensitive to the construction location's limited weather windows or water supply
were used to recommend two types of scheduled discharge rates favouring higher volumes and better water use
efficiencies, respectively. The models were able to capture more than 44 \% of the freezing rate
variations of the traditional AIR. Simulations converting several unscheduled fountains to scheduled ones showed
that at least a three fold increase in water use efficiency is possible without compromising on meltwater
production.

The influence of wind-driven redistribution on the spray radius managed to generate AIRs six times bigger in
spite of temperatures being 3 $\degree C$ warmer. This implies that higher wind speeds drove the volume
differences of AIRs constructed in the Swiss location across three consecutive winters.  However, higher wind
speeds can also cause water losses if water droplets are distributed beyond the spray radius. Therefore, a
critical wind speed needs to be determined in order to force wind-driven redistribution to increase the spray
radius rather than the water losses. Future selection of construction locations and design of automation
algorithms need to capitalise on wind-driven redistribution effects to further increase their water use
efficiency.

Fountain nozzles play an important role in the construction process. First, they consume most of the input water
pressure to form water droplets. Second, their engineering design determines the droplet size distribution and
spray radius. Future research, therefore, must be devoted to engineer fountain nozzles that create water
droplets with a size distribution that requires lesser energy consumption and a trajectory that increases their
spray radius.

\appendix


\section{Model forcing based on water-use efficiency and maximum ice volume objectives} \label{sec:SEB}

The model complexity and data requirement \citep{balasubramanianInfluenceMeteorologicalConditions2022} were
reduced through assumptions that optimise for the ice volume or the water-use efficiency objectives. The
corresponding model assumptions are called IVOM and WEOM respectively. We define the freezing rate and melting
rate as the positive and negative mass change rate, respectively. Assumptions are chosen, based on whether they
overestimate/underestimate the freezing rate. IVOM assumptions overestimates freezing rate whereas WEOM
assumptions underestimates freezing rate. We describe these two kinds of assumptions applied on each of the
energy balance components below: 

\subsection{Surface Area $A_{cone}$ assumptions}

Determination of the surface area during the accumulation period is achieved by assuming a constant ice cone
radius equal to the fountain spray radius. The surface area scales the freezing rate of the AIR. Hence, for the
IVOM version, we assume the maximum possible slope of 1 for the ice cone or in other words $h_{cone} = r_{F}$.
Therefore, area is estimated as:  

\begin{equation} A_{cone} =\sqrt{2} \cdot \pi \cdot r_{F}^2  \end{equation}

Similarly, for the water-use efficiency objective, the area of the conical AIR is approximated to the area of
its circular base. Therefore, area is estimated as:

\begin{equation} A_{cone} =\pi \cdot r_{F}^2  \end{equation}

\subsection{Net shortwave radiation \texorpdfstring{$q_{SW}$}{Lg} assumptions}
\label{sec:SW}

The net shortwave radiation $q_{SW}$ is computed as follows:

\begin{equation} 
q_{SW} = (1- \alpha) \cdot ( SW_{direct} \cdot f_{cone} + SW_{diffuse})
\label{eqn:SW} 
\end{equation}

where $\alpha$ is the albedo value ; $SW_{direct}$ is the direct shortwave radiation; $SW_{diffuse}$ is the
diffuse shortwave radiation and $f_{cone}$ is the solar area fraction.

The data requirement was reduced by estimating the global shortwave radiation and pressure directly using the
location's coordinates and altitude through the solar radiation model described in
\citet{holmgrenPvlibPythonPython2018}. The algorithm used to estimate the clear-sky global radiation is
described in \citet{ineichenBroadbandSimplifiedVersion2008}.  

The diffuse and direct shortwave radiation is determined using the estimated global solar radiation as follows:

\begin{equation}
\begin{split}
  SW_{diffuse} &= cld \cdot SW_{global}\\
  SW_{direct} &= (1-cld) \cdot SW_{global}
\end{split}
\end{equation}

where $cld$ is the cloudiness factor. $cld$ is assumed to be 1 and 0 for the water-use efficiency and ice volume
objective respectively.

We ignore the variations in the albedo and assume it to be equal to snow albedo and ice albedo for the  ice
volume and water-use efficiency objective, respectively.

The solar area fraction $f_{cone}$ of the ice structure exposed to the direct shortwave radiation depends on the
shape considered. It is computed as

\begin{equation}
		f_{cone} =\frac{(0.5 \cdot r_{cone} \cdot h_{cone}) \cdot cos \theta_{sun} +(\pi \cdot
			{(r_{cone})}^2/2) \cdot sin \theta_{sun} }{\pi \cdot r_{cone} \cdot ({(r_{cone})}^2+{(h_{cone})}^2)^{1/2}}\\
\end{equation}

For the ice volume objective, since we assume the slope of the cone to be 1, $f_{cone}$ is determined as follows:

\begin{equation}
		f_{cone} =\frac{ cos \theta_{sun} + \pi \cdot sin \theta_{sun} }{2\sqrt{2} \cdot \pi }
\end{equation}

Similarly, for the water-use efficiency objective, since we assume the slope of the cone to be negligible, we get:

\begin{equation}
		f_{cone} =\frac{ sin \theta_{sun} }{2 }
\end{equation}

\subsection{Net Longwave radiation \texorpdfstring{$q_{LW}$}{Lg} assumptions} 

We assume $T_{ice} = 0 \degree C$ in order to determine outgoing longwave radiation. Since it is challenging to
constrain the minimum ice temperature, we maintain this assumption for both our objectives. However, in order to
estimate atmospheric emissivity, we again assume $cld$ to be 1 and 0 for the water-use efficiency and ice volume
objective respectively.

\subsection{Turbulent fluxes assumptions} \label{sec:Qs}

Turbulent fluxes estimation depend on the slope of the cone through the $\mu_{cone}$ parameter. As suggested 
by \citet{oerlemansBriefCommunicationGrowth2021}, we estimated this parameter as follows:

\begin{equation}
  \mu_{cone} =1 + s_{cone}/2
\end{equation}

Hence, the $\mu_{cone}$ parameter takes values of 1.5 and 1 for the ice volume and water-use efficiency
objective respectively.  Since turbulent fluxes impact both the freezing and the melting rates, this assumption
may not favor the corresponding objectives for certain sites.

\appendixtables   %% needs to be added in front of appendix tables

\begin{table}
  \caption{Free parameters in the model categorised as constant, model hyperparameters and weather 
  parameters with their respective values/ranges.}

	\label{tab:parameters}
	\begin{tabular}{lllll}
		\toprule

		\textbf{Constant Parameters}                       & \textbf{Symbol} & \textbf{Value} &
    \textbf{Unit} & \textbf{References} \\
    Van Karman constant & $\kappa$      & 0.4        &dimensionless & \citet{cuffeyPhysicsGlaciers2010}              \\
    Stefan Boltzmann constant & $\sigma$ & $\num{5.67 e-8} $& $W\, m^{-2}\, K^{-4}$ & \citet{cuffeyPhysicsGlaciers2010}\\
    Air pressure at sea level & $p_{0,a}$ & 1013 & $hPa$  & \citet{molgAblationAssociatedEnergy2004}\\
    Density of water & $\rho_{w}$ & 1000 & $kg\, m^{-3}$    & \citet{cuffeyPhysicsGlaciers2010}\\
    Density of ice & $\rho_{ice}$ & 917 & $kg\, m^{-3}$ & \citet{cuffeyPhysicsGlaciers2010}\\
    Density of air & $\rho_{a}$ &  1.29 & $kg\, m^{-3}$   & \citet{molgAblationAssociatedEnergy2004}\\
    Specific heat of water & $c_{w}$ & 4186 & $J\, kg^{-1}\,\degree C^{-1}$  & \citet{cuffeyPhysicsGlaciers2010}\\
    Specific heat of ice & $c_{ice}$ & 2097 & $J\, kg^{-1}\,\degree C^{-1}$ & \citet{cuffeyPhysicsGlaciers2010}\\
    Specific heat of air & $c_{a}$ & 1010 & $J\, kg^{-1}\,\degree C^{-1}$ & \citet{molgAblationAssociatedEnergy2004}\\
    Thermal conductivity of ice & $k_{ice}$ & 2.123  & $W\, m^{-1}\, K^{-1}$ & \citet{bonalesThermalConductivityIce2017} \\
    Latent Heat of Sublimation & $L_{s}$ & \num{2.848e6}  & $J\, kg^{-1}$ &   \citet{cuffeyPhysicsGlaciers2010}\\
    Latent Heat of Fusion & $L_{f}$ & \num{3.34e5} & $J\, kg^{-1}$ & \citet{cuffeyPhysicsGlaciers2010}\\
    Gravitational acceleration & $g$ & 9.81 & $m\, s^{-2}$ &\citet{cuffeyPhysicsGlaciers2010}\\
    Weather station height & $h_{AWS}$ & 2 & $m$ & assumed \\
    Model timestep                            & $\Delta t$            & $3600$           & $s$ & assumed \\\midrule

		\textbf{Model Hyperparameters} & \textbf{Symbol} & \textbf{Range} & \textbf{Unit} & \textbf{References} \\
    Surface layer thickness             & $\Delta x$            & $[\num{1e-2},\num{1e-1}]$           & $m$ & assumed
    \\\midrule
		\textbf{Weather Parameters} & \textbf{Symbol} & \textbf{Range} & \textbf{Unit} & \textbf{References} \\
    Ice Emissivity                      & $\epsilon_{ice}$      & $[0.95,0.99]$         & dimensionless & \citet{horiInsituMeasuredSpectral2006}             \\
    Surface Roughness                   & $z_0$                 & $[\num{1e-3},\num{5e-3}]$            & $m$  & \citet{brockMeasurementParameterizationAerodynamic2006}       \\
    Ice Albedo                          & $\alpha_{ice}$        & $[0.15,0.35]$         & dimensionless  &
    \citet{steinerModellingIcecliffBackwasting2015};            \\
    & &    &  & \citet{zollesRobustUncertaintyAssessment2019}      \\
    Snow Albedo                         & $\alpha_{snow}$       & $[0.8,0.9]$        & dimensionless  & \citet{zollesRobustUncertaintyAssessment2019}              \\
    Precipitation Temperature threshold & $T_{ppt}$             & $[0,2]$            & $\degree C$& \citet{shichangResponseZhadangGlacier2010}  \\
    Albedo Decay Rate                   & $\tau$                & $[10,22]$           & $days$ &
    \citet{schmidtImportanceAccurateGlacier2017};      \\
    & &    &  & \citet{oerlemansYearRecordGlobal1998}      \\\midrule
	\end{tabular}
\end{table}
\clearpage


\noappendix 

\authorcontribution{
  Suryanarayanan Balasubramanian: Conceptualization, Methodology, Investigation, Data curation, Visualization, Software,
Writing- Original draft preparation

  Martin Hoelzle: Conceptualization, Supervision, Investigation, Writing- Reviewing and Editing 

  Roger Waser: Resources- Automation system, Writing- Reviewing and Editing.} %% this section is mandatory

\competinginterests{The authors declare that they have no known competing financial interests or personal
relationships that could have appeared to influence the work reported in this paper.} %% this section is mandatory even if you declare that no competing interests are present

\begin{acknowledgements}
TEXT
\end{acknowledgements}

\bibliographystyle{copernicus}
\bibliography{zot_refs.bib}

\end{document}
