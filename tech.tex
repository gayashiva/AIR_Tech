\documentclass[utf8]{frontiersSCNS}
\usepackage{gensymb}
\usepackage{url,hyperref,lineno,microtype,subcaption}
\usepackage[onehalfspacing]{setspace}

\linenumbers
\usepackage{wasysym} % provides \DH, \dh, \Thorn, \thorn
% Leave a blank\usepackage{amsmath}
%\DeclareMathOperator{\sign}{sign} line between paragraphs instead of using \\

\usepackage{booktabs}
\usepackage{multirow}
\usepackage{siunitx} %for SI units
\usepackage{lscape} % for landscape table

\def\keyFont{\fontsize{8}{11}\helveticabold }
\def\firstAuthorLast{Balasubramanian {et~al.}} %use et al only if is more than 1 author
\def\Authors{Suryanarayanan Balasubramanian\,$^{1}$, Roger Waser\,$^{2}$, Martin Hoelzle\,$^{1}$}
\def\Address{$^{1}$University of Fribourg, Department of Geosciences, Fribourg, Switzerland $^{2}$University of
Applied Sciences and Arts, Luzern, Switzerland} \def\corrAuthor{Suryanarayanan Balasubramanian}

\def\corrEmail{suryanarayanan.balasubramanian@unifr.ch}


\begin{document}
\onecolumn
\firstpage{1}

\title[Scheduling AIR fountains]{Fountain scheduling strategies to improve water use efficiency of artificial
ice reservoirs (Icestupas)}

\author[\firstAuthorLast ]{\Authors}
\address{}
\correspondance{}

\extraAuth{}

\maketitle

\begin{abstract}
  Artificial Ice Reservoirs (AIRs) are increasingly being used for water storage especially in arid mountain
  regions. Given scarcity of water resources, the water requirement of AIR construction needs to be reduced
  while maximizing meltwater rates. To support improved fountain discharge scheduling, the AIR model was
  selected due to its ability in estimating surface freezing rates in varied climates. To provide for the use of
  the model at the project scale, a new empirical model that computes fountain discharge scheduling requirements
  was developed by incorporating several assumptions into the physical model. Two scheduling modes, namely
  static and dynamic are proposed. During winter of 2021-22, the dynamic scheduling mode was
  compared with a manual mode through the construction of two AIRs in Guttannen, Switzerland. The water
  requirement for the dynamic mode was 87 \% lesser than the manual mode. Ice volume estimates validated with
  drone measurements indicate that the dynamic AIR also achieved higher volumes inspite of the reduced discharge
  rate. However, this volume difference was due to the better design of the dynamic fountain which produced a
  higher spray radius. Overall, our results show that the dynamic and static scheduling mode are attractive
  given their much lower water requirement with much higher WUE.

  \tiny \keyFont{ \section{Keywords:} icestupa, water storage, climate change adaptation,
  geoengineering, nature based solution} 
\end{abstract}

\section{Introduction}
Artificial Ice Reservoirs (AIRs) have recently received much attention in the light of increasing users,
especially in semi-arid and arid mountain regions facing limited water availability and at the same time
increased water needs. However, their water use efficiency (WUE) is very poor. 

Fountain scheduling methods are one option for reducing watering volumes for AIR construction systems and, at
the same time, increase WUE. The goal of fountain scheduling is to make the most efficient use of water by
spraying the right amount of water at the right time, making sure water is available when the AIR surface can
freeze it. Scheduling maximizes AIR WUE by minimizing fountain waste water.  

Proper fountain scheduling requires answers to two questions: (a) When should the water be turned on and off?
(b) How much water should be sprayed? Knowledge of surface freezing rates is important for answering these
questions. Surface freezing rates can be calculated by means of two different approaches: physical
models and empirically based models. The former may be defined as a model in which each of the relevant
energy fluxes at the AIR surface is computed from physically based calculations using direct measurements of the
necessary meteorological variables, and the freezing rate is calculated as the sum of the individual fluxes
scaled with the AIR surface area during the accumulation period. The latter may be defined as a model in which
the freezing rate is calculated based on an assumed linear relationship with air temperature, relative humidity
and wind speed are the measured input variable, although additional input variables, such as incoming solar
radiation, may be incorporated through parametrizations based on time and location. In order to produce
recommended discharge rates, empirical models are preferred due to their parsimony in data requirement and
computing time in comparison with the more sophisticated energy-balance models. But only physical energy balance
models exist for AIRs. Therefore, we present an empirical model that seeks a balance between model complexity
and data availablity.

Since this technology is currently being used by subsistence farmers, we also need to be mindful of the
maintenance and implementation costs. We propose two different modes of fountain scheduling based on operation
costs that can be classified as static or dynamic. In the static mode, the total amount of water
for AIR construction is allocated without specifying its temporal distribution along the accumulation period. By
contrast, in the dynamic mode water is allocated at specific time steps along the accumulation period in
order to achieve even higher WUE.

Both the approaches use a fountain and pipeline system with minimal manual intervention. The dynamic mode is
better suited for locations where the diurnal and seasonal variations in the weather conditions are significant.
In such cases, the real-time adjustments of the discharge rate make efficient fountain scheduling
possible. The static mode is better suited for other locations or for users who cannot afford the additional
hardware requirements to implement the dynamic fountain scheduling mode.


\section{Measurements}
\subsection{The experiment site}
\subsection{Dynamic fountain scheduling prototype}
The automation hardware consists of a weather station, flowmeter, control valve, drain valves, air valves,
fountain, pipeline and a logger. The logger feeds the weather station data to the automation equation and
informs the expected freezing rate to the flowmeter. The flowmeter adjusts the control valve to match the
discharge rate recommendation of the logger. In case this discharge recommendation is below the critical
discharge rate of the the fountain, the drain and air valves empty the water in the pipeline to prevent any
freezing events. Here the critical discharge rate is defined as the discharge rate below which the fountain
pipeline can freeze.

\section{Development and testing of the new empirical model}

The objective of the fountain scheduling algorithm is to estimate the freezing rate given minimal weather,
fountain and location information. For this purpose, we construct a simplified model using components of the
energy balance model used in \cite{balasubramanianInfluenceMeteorologicalConditions2022} and
\cite{oerlemansBriefCommunicationGrowth2021}. We achieve this objective in two steps, namely, by reducing
(a) data requirements and (b) model complexity 

\subsection{Simplified energy balance model}
The growth rate of the AIR can be represented as: 

\begin{equation}
  \frac{\Delta V_{ice}}{\Delta t}  =  \frac{\Delta j_{cone}}{ \Delta t} \cdot A_{cone}
	\label{eqn:freeze}
\end{equation}

where $\frac{\Delta j_{cone}}{\Delta t}$ is the thickness change rate of the AIR which can be represented as: 

\begin{equation}
  \frac{\Delta j_{cone}}{\Delta t}  = \frac{1}{\rho_w} \cdot (\frac{q_{SW} + q_{LW} + q_{S} + q_{F} + q_{G}}{L_F} + \frac{q_{L}}{L_V} )
	\label{eqn:freeze}
\end{equation}

We reduce the data requirements for estimation of each of the energy flux components using assumptions that
underestimate the associated freezing rate in order to optimise for WUE. Further description of our assumptions
are described in the Appendix. Below, we summarise the list of assumptions used:

\begin{itemize}

  \item Shortwave Radiation $q_{SW}$: The direct and diffuse components of shortwave radiation are estimated from the coordinates,
    altitude and time using the PVLIB python package described in \cite{holmgrenPvlibPythonPython2018} . The
    algorithm used to estimate the clear-sky global radiation is described in
    \cite{ineichenBroadbandSimplifiedVersion2008} and the algorithm used to estimate the diffuse part of the
    global radiation is described in \cite{erbsEstimationDiffuseRadiation1982}. The uncertainty of this approach
    is discussed in \cite{ineichenValidationModelsThat2016}. Furthermore, the solar area fraction is also
    approximated by assuming $h_{cone} = r_{cone} = r_{F}$ and the albedo is assumed to be equal to ice albedo
    (0.25). These assumptions underestimate the freezing rate.

  \item Longwave Radiation $q_{LW}$: We assume $T_{ice} = 0 \degree C$. This assumption overestimates $q_{LW}$
    and thereby underestimates the freezing rate.

  \item Turbulent fluxes ($q_{L}$ and $q_{S}$): The software estimates the pressure using the altitude of the
    location and ignores the $\mu_{cone}$ parameter thereby underestimating the turbulent fluxes.

  \item $q_{F}$ and $q_{G}$: We ignore the influence of fountain water heat flux and the ground heat flux
    thereby removing the need for water temperature measurements.

  \item Surface Area $A_{cone}$ : The area of the conical AIR is approximated to the area of its circular base
    produced through the fountain spray radius $r_F$. This assumption is derived from
    \cite{oerlemansBriefCommunicationGrowth2021} model and underestimates the surface area of the AIR during the
    accumulation period and thereby underestimates the freezing rate.

\end{itemize}

\subsection{Surrogate empirical model}

We can now construct a surrogate model that approximates the thickness change rate into a computationally
inexpensive equation with fixed coefficients. We approximate the shortwave radiation flux using a gaussian
function $g($ hour of day $)$ that accounts for its diurnal variation. We approximate the rest of the energy
fluxes by assuming a linear relationship with the weather parameters namely, temperature, humidity, wind speed
and altitude. Using the above simplifications, we can compute the expected thickness change rates using the following set of
equations:

\begin{subequations}
	\begin{align}
		\label{eqn:sun}
  \frac{q_{SW}}{\rho_w \cdot L_F} & \approx \frac{amp}{(\sigma \sqrt{2\pi})} \cdot
  exp\left(\frac{-(hod-\mu)^2}{2\sigma^2}\right) = g(hod)  \\
		\label{eqn:T}
   \frac{q_{LW} + q_{S}}{\rho_w \cdot L_F} + \frac{q_L}{\rho_w \cdot L_V} & \approx a \cdot T_a + b \cdot RH + c \cdot v_a +
  d \cdot alt + e = f(T_a, RH, v_a, alt) \\
		\label{eqn:auto}
  \frac{\Delta j_{cone}}{\Delta t} & \approx g(hod) + f(T_a, RH, v_a, alt)
	\end{align}
\end{subequations}

where $r_F$ is the fountain spray radius and $hod$ is the hour of day.


\begin{table}[ht]
\centering
\caption{}
\label{tab:my-table}
\begin{tabular}{@{}lllllll@{}}
\toprule
Coefficients of & $T_a\, [\degree C]$ & $RH\,[\%]$ & $v_a\, [m\, s^{-1}]$ & $alt\, [km]$  & $constant$ \\ \midrule
Value           & $\num{-5.6 e-3}$     & $\num{-3.2 e-4}$ & $\num{4.7 e-3}$ & $\num{-3.5 e-3}$  & $\num{5.2 e-3}$
                \\ \bottomrule
\end{tabular}
\end{table}


\subsection{Calibration and validation}
\subsection{Dynamic mode strategy}

\section{Results}
\subsection{Model Validation}
\subsubsection{Eddy covariance comparison}
\subsubsection{Surface temperature}
\subsubsection{Ice volume}
\subsection{Model comparison}
\subsubsection{Shortwave radiation comparison}
\subsubsection{Freezing rate comparison}

\begin{figure}[ht]
	\begin{center}
		\includegraphics[width=\linewidth]{Figures/simvsreal_dis.jpg}
	\end{center}
	\caption{Comparison of discharge and ice volume between AIRs produced by dynamic and static fountains. }
	\label{fig:old_icestupa}
\end{figure}


\begin{figure}[ht]
	\begin{center}
		\includegraphics[width=\linewidth]{Figures/autovsman_dis.png}
	\end{center}
	\caption{Comparison of discharge and ice volume between AIRs produced by dynamic and static fountains. }
	\label{fig:old_icestupa}
\end{figure}

\section{Water Use efficiency}

\begin{figure}[ht]
	\begin{center}
		\includegraphics[width=\linewidth]{Figures/wue.png}
	\end{center}
	\caption{Comparison of discharge and ice volume between AIRs produced by dynamic and static fountains. }
	\label{fig:old_icestupa}
\end{figure}

\section{Discussion}
\subsection{Limited and unlimited water availability}

\subsection{Fountain vs weather influence}

\subsection{Ideal fountains and locations for AIR construction}

\subsection{Fountain height and spray radius}


\section{Conclusions}
In this paper, an empirical model that predicts AIR freezing rates is presented. The model was developed and
tested using data collected at Guttannen, Switzerland during the 2021-22 winter season. 

The main purpose of this study was to test the idea that scheduling AIR fountain systems could lead to a
significant improvement in water use efficiency. The empirical fountain scheduling model was calculated and
tested against hourly freezing rates computed by an energy-balance model. 

This study has demonstrated the importance of fountain scheduling. Furthermore, the decrease in performance of
the empirical model has been quantified through comparison of efficiency criteria. The study has demonstrated
that, it is possible to compute more than 90 \% of the freeze rate variations by means of a simplified empirical
model. This model satisfies the need for a less data demanding workflow for scheduling fountains worldwide using
the computation of just 6 coefficents. Such a model will be compatible with the limited data amount that is
typical of any new AIR construction location.

This study has also compared the sensitivity of freezing rates between the weather components and the fountain's
spray radius. It is clear that if improvements are to be achieved, future research must be devoted to modelling
the impact of fountain design on the spray radius.

The model discussed in this paper will be applied worldwide to identify favourable AIR locations in a future
paper.


\section{Appendix}

The area of the conical AIR is approximated to the area of its circular base produced through the fountain spray
radius $r_F$. Therefore, the surface area can be determined using

\begin{equation} A_{cone} =\pi \cdot r_{F}^2 \label{eq:Area} \end{equation}

Admittedly, this assumption underestimates the surface area of the AIR during the accumulation period and
thereby underestimates the freezing rate.

We approximate the energy balance at the surface of an AIR by a one-dimensional description of energy fluxes as
used in \cite{balasubramanianInfluenceMeteorologicalConditions2022}:


Upward and downward fluxes relative to the ice surface are positive and negative, respectively. The first
term represents the energy change used for freezing the fountain water and melting the ice respectively.
$q_{SW}$ is the net shortwave radiation; $q_{LW}$ is the net longwave radiation; $q_{L}$ and $q_{S}$ are the
turbulent latent and sensible heat fluxes. 

The software assumes $T_{ice} = 0 \degree C$ and therefore ignores the temperature change flux $q_{T}$, fountain
heat flux $q_{F}$ and ground heat flux $q_{G}$. All these assumptions overestimate the freezing rate.

Furthermore, we the rest of the energy balance components based on their air temperature and solar
radiation dependence. Namely, $q_{LW}$, $q_{L}$ and $q_{S}$ contribute to temperature induced freeze rate and
$q_{SW}$ contributes to solar radiation induced melt rate during the accumulation period.  Particularly,

For the static method, we only need to determine the night freezing rates whereas for the dynamic approach we
also need to account for the day melt during the accumulation period.

\subsection{Net Longwave radiation \texorpdfstring{$q_{LW}$}{Lg}} \label{sec:LW}
The net longwave radiation $q_{LW}$ is determined as follows:

\begin{equation}
	q_{LW}= \sigma \cdot \epsilon_a \cdot {(T_a+ 273.15)}^4 -\sigma \cdot \epsilon_{ice} \cdot {(T_{ice}+ 273.15)}^4
	\label{eqn:LW}
\end{equation}

where $T_a$ represents the measured air temperature, $\epsilon_a$ denotes the atmospheric emissivity $T_{ice}$
is the modelled surface temperature given in [$\degree C$], $\sigma=5.67\cdot10^{-8}\,Jm^{-2}s^{-1}K^{-4}$ is
the Stefan-Boltzmann constant and $\epsilon_{ice}$ is the corresponding emissivity value for the Icestupa
surface (0.97).

We approximate the atmospheric emissivity $\epsilon_a$ ,
considering air temperature and vapor pressure (Eqn. \ref{eqn:atm_e}). The vapor pressure of air over water and
ice was obtained using Eqn. \ref{eqn:vp}.  The expression defined in \cite{brutsaertDerivableFormulaLongwave1975} for clear skies
(first term in equation \ref{eqn:atm_e}) is extended with the correction for cloudy skies after
\cite{brutsaertEvaporationAtmosphereTheory1982} as follows:

\begin{equation}
	\epsilon_a=1.24 \cdot (\frac{p_{v,w}}{(T_a+273.15)})^{1/7}\cdot(1+0.22\cdot{cld}^2) \label{eqn:atm_e}
\end{equation}

with a cloudiness index $cld$, ranging from 0 for clear skies to 1 for complete overcast skies. 

The software assumes $T_{ice} = 0 \degree C$. This assumption overestimates $q_{LW}$ and thereby underestimates
the freezing rate.

\subsection{Turbulent fluxes} \label{sec:Qs}

The turbulent sensible $q_{S}$ and latent heat $q_{L}$ fluxes are computed with the following expressions
proposed by \cite{garrattAtmosphericBoundaryLayer1992}:

\begin{equation}
	q_{S}= c_{a} \cdot \rho_{a} \cdot p_{a}/p_{0,a} \cdot \frac{\kappa^2 \cdot v_a \cdot
		(T_a-T_{ice})}{{(\ln{\frac{h_{AWS}}{z_{0}}})}^2}
	\label{eqn:qs}
\end{equation}

\begin{equation}
	q_{L}= 0.623 \cdot L_s \cdot \rho_{a}/p_{0,a} \cdot \frac{\kappa^2 \cdot
	v_a(p_{v,w}-p_{v,ice})}{{(\ln{\frac{h_{AWS}}{z_{0}}})}^2}
\end{equation}

where $h_{AWS}$ is the measurement height above the ground surface of the AWS (around $2\,m$ for all sites),
$v_a$ is the wind speed in [$m\,s^{-1}$], $c_a$ is the specific heat of air at constant pressure (1010 J
$kg^{-1} K^{-1}$), $\rho_{a}$ is the air density at standard sea level (1.29 $kg m^{-3}$), $p_{0,a}$ is the air
pressure at standard sea level (1013 $hPa$), $p_{a}$ is the measured air pressure, $\kappa$ is the von Karman
constant (0.4), $z_{0}$ is the surface roughness (3 $mm$) and $L_s$ is the heat of sublimation (2848
$kJ\,kg^{-1}$).  The vapor pressure of air with respect to water ($p_{v,w}$) and with respect to ice
($p_{v,ice}$) was obtained using the formulation given in \cite{huangSimpleAccurateFormula2018} :

\begin{equation}
	\begin{split}
		p_{v,w}&=e^{\frac{(34.494 - \frac{4924.99}{T_{a} + 237.1})}{(T_a + 105)^{1.57} \cdot 100}} \cdot \frac{RH}{100} \\
		p_{v,ice}&=e^{\frac{(43.494 - \frac{6545.89}{T_{ice} + 278})}{(T_{ice} + 868)^{2} \cdot 100}} \\
	\end{split} \label{eqn:vp}
\end{equation}

The software ignores the $\mu_{cone}$ parameter thereby underestimating the turbulent fluxes. Since turbulent
fluxes impact both the freezing and the melting rates, this assumption may not underestimate freezing rates.

\subsection{Net shortwave radiation \texorpdfstring{$q_{SW}$}{Lg}}
\label{sec:SW}

The net shortwave radiation $q_{SW}$ is computed as follows:

\begin{equation} q_{SW} = (1- \alpha_{ice}) \cdot ( SW_{direct} \cdot f_{cone} + SW_{diffuse})
\label{eqn:SW} \end{equation}

where $\alpha_{ice}$ is the bare ice albedo value (0.25); $SW_{direct}$ is the direct shortwave radiation. The
global shortwave radiation used is modelled using the parametrisation proposed by \cite{woolfComputationSolarElevation1968}.

The solar area fraction $f_{cone}$ of the ice structure exposed to the direct shortwave radiation depends on the
shape considered. Using the solar elevation angle $\theta_{sun}$, the solar beam can be considered to have a
vertical component, impinging on the horizontal surface (semicircular base of the AIR), and a horizontal
component impinging on the vertical cross section (a triangle). The solar elevation angle $\theta_{sun}$ used is
modelled using the parametrisation proposed by \cite{woolfComputationSolarElevation1968}. Here we overestimate the impact of direct
solar radiation by assuming $h_{cone} = r_{cone} = r_{F}$. Accordingly, $f_{cone}$ is determined as follows:

\begin{equation}
	\begin{split}
		f_{cone}& =\frac{ cos \theta_{sun} + \pi \cdot sin \theta_{sun} }{2\sqrt{2} \cdot \pi }\\
	\end{split}
	\label{ eqn:f_{cone}}
\end{equation}

The software ignores the variations in the albedo and assumes it to be equal to that of ice to simplify the
model. This assumption overestimates the solar radiation absorption thereby underestimating the freezing rate.


\bibliographystyle{frontiersinSCNS_ENG_HUMS} \bibliography{zot_refs}

\end{document}

% \begin{landscape}
% \begin{table}[]
% \centering
% \caption{}
% \label{tab:my-table}
% \begin{tabular}{@{}lllll@{}}
% \toprule
% \textbf{Module name} & \textbf{Symbol} & \textbf{Full eqn} & \textbf{Simplified eqn} & \textbf{Assumptions} \\ \midrule
% \multicolumn{1}{|l}{Surface Area}        & $A_{cone}$ &  & \pi \cdot r_{F}^2 & \multicolumn{1}{l|}{} \\ \midrule
% \multicolumn{1}{|l}{Shortwave Radiation} & $q_{SW}$ &  & (1- \alpha_{ice}) \cdot ( (1- cld) \cdot SW_{global} \cdot f_{cone} + cld \cdot SW_{global}) & \multicolumn{1}{l|}{} \\ \midrule
% \multicolumn{1}{|l}{Longwave Radiation}  & $q_{LW}$ &  & \sigma \cdot \epsilon_a \cdot {(T_a+ 273.15)}^4 -\sigma \cdot \epsilon_{ice} \cdot {273.15}^4 & \multicolumn{1}{l|}{} \\ \midrule
% \multicolumn{1}{|l}{Sensible Heat}       & $q_{S}$ &  &  & \multicolumn{1}{l|}{} \\ \midrule
% \multicolumn{1}{|l}{Latent Heat}         & $q_{L}$ &  &  & \multicolumn{1}{l|}{} \\ \bottomrule
% \multicolumn{1}{|l}{Temperature heat flux} & $q_{T}$ &  & 0 & \multicolumn{1}{l|}{} \\ \bottomrule
% \multicolumn{1}{|l}{Fountain discharge heat flux} & $q_{F}$ &  & 0 & \multicolumn{1}{l|}{} \\ \bottomrule
% \multicolumn{1}{|l}{Ground heat flux}    & $q_{G}$ &  & 0 & \multicolumn{1}{l|}{} \\ \bottomrule
% \end{tabular}
% \end{table}
% \end{landscape}
