\documentclass[utf8]{frontiersSCNS}
\usepackage{gensymb}
\usepackage{url,hyperref,lineno,microtype,subcaption}
\usepackage[onehalfspacing]{setspace}

\usepackage{tabularx}
\linenumbers
\DeclareUnicodeCharacter{0301}{}
\DeclareUnicodeCharacter{2212}{}
\usepackage{wasysym} % provides \DH, \dh, \Thorn, \thorn
% Leave a blank\usepackage{amsmath}
%\DeclareMathOperator{\sign}{sign} line between paragraphs instead of using \\

% \usepackage{csvsimple} % for csv tables
\usepackage{booktabs}
\usepackage{multirow}
\usepackage{siunitx} %for SI units
\usepackage{tabularx}

\def\keyFont{\fontsize{8}{11}\helveticabold }
\def\firstAuthorLast{Balasubramanian {et~al.}} %use et al only if is more than 1 author
\def\Authors{Suryanarayanan Balasubramanian\,$^{1*,2}$, Martin Hoelzle\,$^{1}$, Michael Lehning\,$^{3}$, Jordi
	Bolibar\,$^{4}$, Sonam Wangchuk\,$^{2}$, Johannes Oerlemans\,$^{4}$ and Felix Keller\,$^{5,6}$}
\def\Address{$^{1}$University of Fribourg, Department of Geosciences, Fribourg, Switzerland\\ $^{3}$WSL Institute for Snow and Avalanche
	Research, Davos, Switzerland\\ $^{2}$Himalayan Institute of Alternatives Ladakh, Leh, India\\ $^{4}$Institute
	for Marine and Atmospheric Research, Utrecht University, Utrecht, The Netherlands\\ $^{5}$Academia Engiadina,
	Samedan, Switzerland\\ $^{6}$ETH, Zürich, Switzerland} \def\corrAuthor{Suryanarayanan Balasubramanian}

\def\corrEmail{suryanarayanan.balasubramanian@unifr.ch}


\begin{document}
\onecolumn
\firstpage{1}

\title[Artificial Ice Reservoirs]{Influence of meteorological conditions on artificial ice reservoir (Icestupa) evolution}

\author[\firstAuthorLast ]{\Authors}
\address{}
\correspondance{}

\extraAuth{}

\maketitle

\begin{abstract}

  Since 2014, mountain communities in Ladakh, India have been constructing dozens of Artificial Ice Reservoirs
  (AIRs) by spraying water through fountain systems every winter. The meltwater from these structures is
  crucial to meet irrigation water demands during spring. However, there is a large variability associated with
  this water supply due to the local weather influences at the chosen location. This study compared the ice
  volume evolution of an AIR built in Ladakh, India with two others built in Guttannen, Switzerland using a
  surface energy balance model.  Model input consisted of meteorological data in conjunction with fountain
  discharge rate (mass input of an AIR). Model calibration and validation were completed using ice volume and
  surface area measurements taken from several drone surveys. The model was successful in estimating the
  observed ice volume evolution with a root mean square error within $18 \%$ of the maximum ice volume for all
  the AIRs. The location in Ladakh had a maximum ice volume four times larger compared to the Guttannen site.
  However, the corresponding water losses for all the AIRs were more than three-quarters of the total fountain
  discharge due to high fountain wastewater. Drier and colder locations in relatively cloud-free regions are
  expected to produce long-lasting AIRs with higher maximum ice volumes. This is a promising result for dry
  mountain regions, where AIR technology could provide a relatively affordable and sustainable strategy to
  mitigate climate change induced water stress.

	\tiny
	\keyFont{ \section{Keywords:} icestupa, water storage, climate change adaptation, geoengineering } %All article types: you may provide up to 8 keywords; at least 5 are mandatory.
\end{abstract}

\section{Introduction}


\section{Study Sites and data}


\subsection{Meteorological data}



\subsection{Fountain observations}


\subsection{Drone surveys}


\section{Model setup}




\section{Results}


\subsection{Validation}


\section{Discussion}


\section{Conclusions}


\section{Appendix}


\section*{Conflict of Interest Statement} The authors declare that the research was conducted in the absence of
any commercial or financial relationships that could be construed as a potential conflict of interest.

\section*{Author Contributions} SB, MH, SW and FK designed the study.  SB developed the methodology with inputs
from MH.  MH, ML and JO reviewed the algorithm and helped improve it. SB processed the drone data. SB wrote the
model code. JB helped with model validation and uncertainty assessment. SB, MH, FK and SW participated in the
fieldwork.  SB led the writing of the paper and all co-authors contributed to it.

\section*{Funding} This work was supported and funded by the University of Fribourg and by the Swiss Government
Excellence Scholarship (SB). The associated fieldwork in India was supported by Himalayan Institute of
Alternatives and funded by the Swiss Polar Institute.

\section*{Acknowledgments} This work would not have been possible without the untiring efforts of the Swiss and
Indian icestupa construction teams through the winters of 2019, 2020 and 2021. We thank Mr. Adolf Kaeser and Mr.
Flavio Catillaz from Eispalast Schwarzsee (CH19); Daniel Beurki from the Guttannen Bewegt Association (CH20 and
CH21); Norboo Thinles, Nishant Tiku, Sourabh Maheshwari and the rest of the HIAL team (IN21).  We would also
like to thank Hansueli Gubler for designing the Swiss AWS; Dr. Tom Matthews for designing the Indian AWS;
Michelle Stirnimann for conducting the CH20 drone surveys and Digmesa AG for subsidising their flowmeter used in
the experiment.  We would particularly like to thank Prof. Thomas Schuler and 4 anonymous reviewers who gave us
important inputs to improve the paper. We also thank Prof. Christian Hauck, Prof. Nanna B. Karlsson Dr. Andrew
Tedstone and Alizé Carrère for valuable suggestions that improved the manuscript.

\section*{Data Availability Statement} Model code is freely available in GitHub
(\url{https://github.com/Gayashiva/air_model}, last access: 17 December 2021) for non-profit purposes. The drone
data can be obtained from the authors upon request.

\bibliographystyle{frontiersinSCNS_ENG_HUMS} \bibliography{references}

\end{document}
